\documentclass[a4paper,oneside,leqno,12pt]{book}
\usepackage[a4paper,left=4cm,right=3cm,top=3cm,bottom=3cm,head=0cm,headsep=0cm]{geometry}
\usepackage[utf8]{inputenc}
\usepackage[T1]{fontenc}
\usepackage{amsmath}
\usepackage{amssymb}
\usepackage{amsthm}
\usepackage{graphicx}
\usepackage{epstopdf}
\usepackage{inputenc}
\usepackage{mathtools}
\usepackage{geometry}
\usepackage{polski}
\usepackage{float}
\usepackage{epstopdf}
\usepackage{caption}
\usepackage{subcaption}
\usepackage{color}
\usepackage{url}
\usepackage{enumitem}
\usepackage{multicol}
\usepackage{pdfpages}
\usepackage{import}

%\usepackage{algorithmicx}
\usepackage{algorithm}
\usepackage{algpseudocode}
\usepackage{caption}% http://ctan.org/pkg/caption
%\usepackage{algpascal}

\usepackage{pbox}
\usepackage{array}

%\usepackage{listings}
\usepackage[outputdir="temp"]{minted}

\usepackage{hyperref}

\usepackage{appendix}

\usepackage{float}
\floatstyle{plaintop}
\restylefloat{table}
%\usepackage[tableposition=top]{caption} %table titles on top

\subimport{includes/}{def.tex}

\begin{document}

\subimport{includes/}{strona_tytulowa.tex}
\tableofcontents

\chapter*{Wstęp}
\addcontentsline{toc}{chapter}{Wstęp}
\subimport{includes/}{ch0_wstep.tex}

\chapter{Wiadomości wstępne}\label{ch:basics}
\subimport{includes/}{ch1_ustalenia_wstepne.tex}

\chapter{Mnożenie macierzy}\label{ch:matrix_multiplication}
\subimport{includes/}{ch2_algorytmy_rownolegle.tex}

\chapter{Rezultaty doświadczalne}\label{ch:performance}
\subimport{includes/}{ch3_doswiadczalne}

\begin{appendices}
\chapter{Implementacja}\label{app:code}
\section{Opis programu}
\subimport{includes/}{app_opis_programu}
	\subsection{Przebieg testów na klastrze Solaris}
	\subimport{includes/}{app_przebieg_testow}
	
\section{Omówienie kodu źródłowego}\label{app:improvements}
\subimport{includes/}{app_omowienie_kodu_zrodlowego}
\chapter{Załącznik}
\subimport{includes/}{app_zalacznik}
\end{appendices}
\bibliographystyle{unsrt}
\bibliography{bibdb}

\end{document}