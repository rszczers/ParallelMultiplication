\documentclass[a4paper,oneside,leqno,12pt]{book}
\usepackage[a4paper,left=4cm,right=3cm,top=3cm,bottom=3cm,head=0cm,headsep=0cm]{geometry}
\usepackage[utf8]{inputenc}
\usepackage{amsmath}
\usepackage{amssymb}
\usepackage{amsthm}
\usepackage{graphicx}
\usepackage{epstopdf}
\usepackage{inputenc}
\usepackage{mathtools}
\usepackage{geometry}
\usepackage{polski}
\usepackage{float}
\usepackage{graphicx}
\usepackage{epstopdf}
\usepackage{subcaption}
\usepackage{color}
\usepackage{url}
\usepackage{clrscode3e}
\usepackage{algorithmicx}
\usepackage[]{algorithm2e}
\usepackage{enumitem}
\usepackage{multicol}
\usepackage{pdfpages}
\usepackage{import}


\subimport{includes/}{def.tex}

\begin{document}
\subimport{includes/}{strona_tytulowa.tex}
\tableofcontents

\chapter{Wstęp}
\subimport{includes/}{ch_wstep.tex}
\chapter{Wiadomości wstępne}
\subimport{includes/}{ch_ustalenia_wstepne.tex}
\chapter{Klasyczne algorytmy mnożenia macierzy}
\subimport{includes/}{ch_algorytmy_klasyczne.tex}
\chapter{Równoległe algorytmy mnożenia macierzy}
\subimport{includes/}{ch_algorytmy_rownolegle.tex}

\chapter{Interfejsy programowania równoległego}
\section{MPI}
\section{OpenMP}
\section{Cilk++}

\chapter{Implementacja wybranych algorytmów}
\section{MPI}
\section{OpenMP}
\section{Cilk++}

\bibliographystyle{unsrt}
\bibliography{bibdb}

\end{document}