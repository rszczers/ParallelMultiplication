\subsection{Wprowadzenie}
W ramach pracy licencjackiej zaimplementowano w języku C prosty system na potrzeby testowania wybranych algorytmów obliczania iloczynu macierzowego. Program wykorzystuje interfejsy programowania równoległego MPI i OpenMP oraz bibliotekę matematyczną Intel MKL. Użyto interfejsu argp do parsingu argumentów wejściowych oraz wyświetlania opcji \texttt{--help} i \texttt{--version} w stylu GNU (listing \ref{ls:pmm_help}). Możliwe jest wykonanie:
\begin{enumerate}
\item Naiwnego algorytmu sekwencyjnego (opcja \texttt{--method=SEQUENTIAL}),
\item Mnożenia z operacją \texttt{parfor} (opcja \texttt{--method=OMP}),
\item Mnożenia z wykorzystaniem operacji \texttt{cblas\_dgemm} z biblioteki Intel MKL (opcja \texttt{--method=MKL})
\item Klasycznego algorytmu Cannona (opcja \texttt{--method=CANNON}),
\item Algorytmów hybrydowych:
\begin{enumerate}
\item Algorytmu Cannona z drobnoziarnistą operacją \texttt{parfor} \\ (opcja \texttt{--method=CANNON\_OMP}), 
\item Algorytmu Cannona z drobnoziarnistą operacją \texttt{cblas\_dgemm} z biblioteki Intel MKL, (opcja \texttt{--method=CANNON\_DGEMM}).
\end{enumerate}
\end{enumerate}

Program \texttt{gen} powstał na potrzeby szybkiego generowania dużych macierzy gęstych w zapisie wierszowym (patrz \ref{sub:datatypes}). Wygenerowane przez program pliki składają się z ciągów liczb pseudolosowych o pewnych zadanych własnościach (patrz listing \ref{ls:pmm_help}). 


\subsection{Testowanie}
Na potrzeby pracy proces testowania został całkowicie zautomatyzowany ze względu przeprowadzenia dużej liczby testów dla różnych grup danych wejściowych i na potrzeby graficznej prezentacji wydajności obliczeń dla różnej ilości procesów i wątków. Całość zarządzania jest zestawem domyślnie zdefiniowanych celów zdefiniowanych w pliku \texttt{Makefile} i realizowanych przez narzędzie Make. Najważniejsze z nich to:
\vspace{5mm}


\noindent\begin{tabular}{p{4cm}p{9cm}}
\texttt{make} & Odbudowuje drzewo katalogów i kompiluje programy do katalogu \texttt{./build/}. \\
\texttt{make data} & Generuje przykładowe macierze kwadratowe \texttt{\$(PATH\_A)} i \texttt{\$(PATH\_B)} o szerokości \texttt{\$(SIZE)}. \\
\texttt{make seq} & Wykonanie mnożenia algorytmem sekwencyjnym. \\
\texttt{make omp} & Wykonanie mnożenia z operacją \texttt{parfor}. \\
\texttt{make mkl} & Wykonanie mnożenia z operacją \texttt{cblas\_dgemm}. \\
\texttt{make cannon} & Wykonanie algorytm Cannona.\\
\texttt{make cannon\_omp} & Wykonanie algorytm Cannona z operacją \texttt{parfor}.\\ 
\texttt{make cannon\_dgemm} & wykonuje algorytm Cannona z operacją MKL \texttt{cblas\_dgemm}
\end{tabular}


\vspace{5mm}
Dla obliczeń równoległych istotne są dwie zmienne: \texttt{NPROCS} oraz \texttt{OMP\_THR\-EADS}. Pierwsza z nich określa liczbę procesów do uruchomienia, druga --- liczbę wątków.

Przy domyślnych ustawieniach po wykonaniu każdego zadania w katalogu \texttt{./debug/} tworzony jest plik \texttt{debug\_X}, gdzie \texttt{X} to data wykonania zadania w formacie uniksowym. Zawiera on dane o czasie wykonania programu, jego części sekwencyjnej, rozmiarze danych, liczbie procesów i wątków. Opcja \texttt{-s} pozwala na zapis danych częściowych każdego z procesów na każdym etapie wykonywania algorytmu. Pliki zapisywane są w czytelnej postawi do katalogu \texttt{./debug/} pod nazwą \texttt{XXXX\_YYYY\_Z}, gdzie \texttt{XXXX} identyfikuje proces z którego zapisano dane, \texttt{YYYY} --- krok algorytmu, \texttt{Z} --- macierz \(A\), \(B\) lub \(C\), której fragment zapisano. Pozwala to łatwo prześledzić wszystkie etapy obliczeń.




% Zestaw skryptów w językach Perl i Bash analizują pliki wynikowe z katalogu ./debug/ i przetwarzają je na pliki danych programu Gnuplot. Odpowiednie wykresy generują się po wykonaniu testów.


% Program jest elastyczny. Pozwala użytkownikowi mnożyć dowolne dwie wybrane macierze zapisane w pliku w formacie wierszowym i zapisać wynik w wybranym przez użytkownika położeniu. 




% Implementacja algorytmu Cannona, chociaż sam algorytm w oryginalnej wersji pracuje tylko na macierzach kwadratowych o rozmiarze wielokrotności szerokości lub długości siatki procesow, pozwala pracować na macierzach dowolnych rozmiarów. Program przed wykonaniem obliczeń skaluje macierze do wymiarów wymaganych przez wybrany algorytm. 


Program rozwijany jest w serwisie github pod adresem \url{http://github.com/rszczers/ParallelMultiplication}.

\begin{listing}[h]
\usemintedstyle{vim}
\begin{minted}[fontsize=\footnotesize]{console}
rszczers@solaris:~/ParallelMultiplication/build$ mpirun -np 1 ./pmm --help
Usage: pmm [OPTION...] -A matrixA_path -B matrixB_path -m NUM -k NUM -n NUM
Parallel Matrix Multiplication.

  -a, --method=METHOD        Algorithm used
  -A, --inputA=FILE          Path to input FILE containing matrix A data
  -B, --inputB=FILE          Path to input FILE containing matrix B data
  -d, --debug[=DIR]          Path to debug directory
  -k NUM                     Number of rows of B
  -l, --list                 Show list of available algorithms
  -m NUM                     Number of rows of A
  -n NUM                     Number of columns of B
  -o, -C, --output[=FILE]    Path to output FILE containing matrix C=A*B data
  -q, --quiet                Do not show any computations
  -s, --steps                Dump data from each node for every step
  -t, --time                 Show elapsed time
  -v, --verbose              Show all computations
  -?, --help                 Give this help list
      --usage                Give a short usage message
  -V, --version              Print program version

Mandatory or optional arguments to long options are also mandatory or optional
for any corresponding short options.

Report bugs to <rafal.szczerski@gmail.com>.
\end{minted}
\begin{minted}[fontsize=\footnotesize]{console}
rszczers@solaris:~/ParallelMultiplication/build$ ./gen --help
Usage: gen [OPTION...] -l NUM

  -f, --float                Generate floating point numbers
  -l, --length=NUM           Lenght of array to generate
  -m, --min=NUM              Lower boundary of generated elements
  -M, --max=NUM              Upper boundary of generated elements
  -p, --path[=FILE]          Output file
  -v, --verbose              Verbose mode
  -?, --help                 Give this help list
      --usage                Give a short usage message

Mandatory or optional arguments to long options are also mandatory or optional
for any corresponding short options.
\end{minted}
\caption{Ekran pomocy programów \texttt{pmm} i \texttt{gen}.}
\label{ls:pmm_help}
\end{listing}