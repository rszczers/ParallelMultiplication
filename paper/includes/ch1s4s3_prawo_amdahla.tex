W złożoności \(T_{p}(n)\) można wyróżnić operacje obliczeniowe, które muszą być wykonane sekwencyjnie, \(T^{s}_{1}(n)\), oraz obliczenia, które mogą być wykonane równolegle, \(T^{r}_{1}(n)\). Inaczej:
\begin{align}
T_{1}(n) = T^{s}_{1}(n) + T^{r}_{1}(n)
\end{align}
Zakładając, że obliczenia \(T^{r}(n)\) da się równomiernie rozdzielić między \(p\) procesorami, przyspieszenie \(S(p, n)\) wynosi wówczas
\begin{align}\label{eq:supSpn}
S_{p}(n) = \frac{T_{1}(n)}{T_{p}(n)}\leq\frac{T^{s}_{1}(n) + T^{r}_{1}(n)}{T^{s}_{1}(n) + T^{r}_{1}(n)/p + T^{o}_{p}(n)}
\end{align}
gdzie \(T^{o}_{p}(n)\) jest złożonością dodatkową wynikającą z organizacji obliczeń równoległych.


Rozważmy teraz algorytm sekwencyjny o złożoności \(T_{1}(n)\) rozwiązujący zadany problem \(P\) o ustalonym rozmiarze \(n\). Niech \(s\) oznacza część operacji algorytmu, która musi być wykonana sekwencyjnie, zaś \(r\) część operacji, która może być wykonana równolegle. Oznaczmy: \(T^{s}(n) = sT_{1}(n)\), \(T^{r}(n)=rT_{1}(n)\), gdzie \(s+r=1\). 


Przyspieszenie algorytmu uzyskane po jego zrównolegleniu można wyznaczyć upraszczając wzór \eqref{eq:supSpn} przez pominięcie złożoności \(T^{o}_{p}(n)\). Mamy wówczas:

\begin{equation}
\begin{split}\label{eq:amdahl}
S_{p}(n) &= \frac{T_{1}(n)}{T_{p}(n)}\leq\\
&\leq \frac{T^{s}_{1}(n) + T^{r}_{1}(n)}{T^{s}_{1}(n) + T^{r}_{1}(n)/p + T^{o}_{p}(n)}\leq\\
&\leq \frac{sT_{1}(n) + rT_{1}(n)}{sT_{1}(n) + rT_{1}(n)/p} = \frac{s+r}{s+r/p} = \frac{1}{s+r/p}= \\
&= \left(s+\frac{1-s}{p}\right)^{-1}
\end{split}
\end{equation}
gdzie \(s\) – część obliczeń w algorytmie które muszą być wykonane sekwencyjnie; \(p\) – liczba procesorów.\\
Otrzymany wzór \eqref{eq:amdahl} nazywamy \textbf{prawem Amdahla}. 


\begin{definicja}[Prawo Amdahla\cite{Quinn}]\label{def:prawo_amdahla}
Niech \(s\) będzie częścią operacji w algorytmie \(R\), która musi być wykonana sekwencyjnie, taką że \(0\leq s \leq 1\). Wówczas maksymalne przyspieszenie \(\Psi\) osiągalne przez komputer równoległy z \(p\) procesorami wykonujący algorytm \(R\) spełnia nierówność:
\begin{align}
\Psi_{p}(n) \leq \frac{1}{s+(1-s)/p}
\end{align}
\end{definicja}
\begin{uwaga}
Nierówność z definicji \ref{def:prawo_amdahla} służy do wyznaczania górnego ograniczenia przyspieszenia będącego funkcją wielkości \(s\) oraz liczby procesorów \(p\) przy ustalonym rozmiarze problemu \(n\).
\end{uwaga}

\begin{przyklad}
Przypuśćmy, że dysponujemy sześciordzeniowym procesorem i chcemy ocenić czy warto szukać równoległej wersji programu dla rozwiązania pewnego problemu. Ustaliliśmy, że 90\% czasu wykonania programu przeznacza się na wykonanie pewnej funkcji, którą chcemy zrównoleglić. Pozostałe 10\% czasu wykonania zajmują funkcję, które musimy wykonywać na jednym procesorze. Chcemy ocenić największe przyspieszenie jakiego możemy się spodziewać po równoległej wersji naszego programu. W tym celu możemy skorzystać z prawa Amdahla. Mamy:
\begin{align*}
S_{p}(n) \leq \frac{1}{0.1 + (1-0.1)/6} = 4.
\end{align*}
\noindent Powinniśmy się zatem spodziewać przyspieszenia o wartości co najwyżej 4.
\end{przyklad}


\begin{wniosek}
Przechodząć z wyrażeniem \eqref{eq:amdahl} do granicy \(p\to\infty\) mamy \(\lim_{p\to\infty}\frac{1}{s+(1-s)/p} = \frac{1}{s}\). Widać, że maksymalne przyspieszenie \(S_{p}(n)\), jakie można osiągnać nie zależy od liczby użytch procesorów \(p\), ale od ilości obliczeń sekwencyjnych \(s\) (pomijając dodatkowe koszty ogranizacji obliczeń).
\end{wniosek}

\begin{przyklad}
Przypuśćmy, że 25\% operacji w algorytmie równoległym musi być wykonanych równolegle. Wówczas maksymalne osiągalne przyspieszenie przy przeprowadzaniu obliczeń na coraz większej liczbie procesorów wynosi:
\begin{align*}
\lim_{p\to \infty} \frac{1}{0.25+(1-0.25)/p} = 4.
\end{align*}
\end{przyklad}

\begin{przyklad}
Powiedzmy, że zaimplementowaliśmy równoległą wersję pewnego algorytmu sekwencyjnego o złożoności czasowej \(\Theta(n^2)\), gdzie \(n\) to rozmiar danych.
Sekwencyjna część programu potrzebuje do samego załadowania danych oraz wyświetlenia wyniku obliczeń czasu \((18000+n) \; \mu s\). Część obliczeniowa jest równoległa i potrzebuje czasu \((n^2/100)\;\mu s\). Dla danych o rozmiarze \(10\,000\) stosując prawo Amdahla możemy ocenić maksymalne przyspieszenie jakie uzyskamy rozwiązując problem równolegle:
\begin{align*}
S_{p}(n) \leq \frac{(28\,000 + 1\,000\,000) \mu s}{(28\,000 + 1\,000\,000/p) \mu s}
\end{align*}

\end{przyklad}
