% \subsubsection{Algorytm sumowania}
% Niech dany będzie tablica \(A\) \(n=2^k\) liczb i maszyna PRAM z n-procesorami \(\{P_1, P_2, \dots, P_n\}\). Każdy z procesorów wykonuje synchronicznie poniższy algorytm

% \label{alg:crew_pram_sum}
% % \begin{algorithm}[H]
% % \SetKwFunction{Gread}{global read}
% % \SetKwFunction{Gwrite}{global write}
% % \SetKwInOut{Input}{Dane wejściowe}\SetKwInOut{Output}{Dane wyjściowe}
% % \Input{Tablica \(A\) długości \(n=2^k\) przechowywana w pamięci wspólnej. Każdy procesor ma zainicjalizowane zmienne lokalne \(n\) oraz identyfikator \(i\)}
% % \Output{Suma \(S\) wartości tablicy \(A\). Tablica \(A\) nie ulega zmianie}
% % \begin{enumerate}
% %  \item \Gread{A(i), a}
% %  \item \Gwrite(a, B(i))
% %  \item for \(h = 1\) to \( \log{n}\) do\\
% % 	 if \(i \leq n/2^k\) then\\
% % 	 \Gread{B(2i-1), x}\\
% % 	 \Gread{B(2i),y}\\
% % 	 Set z:= x + y\\
% % 	 \Gwrite(z, B(i))\\
% %  \item \(if(i=1)\) then \Gwrite{z, S}
% % \end{enumerate}
% % \caption{Algorytm sumowania w PRAM\label{alg:pram_sum}}
% % \end{algorithm}

% Przypadek dla \(n=8\) ilustruje rysunek \ref{fig:pram_sum}. W pierwszym i drugim kroku kopia B tablicy A jest tworzona w pamięci wspólnej. 
% Zadania obliczeniowe w kroku 3 są na podstawie wyważonego drzewa binarnego, którego liście odpowiadają elementom tablicy A. Procesor odpowiedzialny za wykonanie za wykonanie operacji jest określony przez indeks poniżej węzła reprezentującego tę operację. Zauważmy, że procesor \(P_1\), odpowiedzialny za ustawianie wartości \(B(1)\) i zapisywanie sumy \(S\), jest zawsze aktywny w trakcie wykonywania algorytmu, podczas gdy procesory \(P_5, P_6, P_7, P_8\) są aktywne tylko podczas kroków 1 i 2.

% \begin{uwaga}
% Pomijamy szczegóły operacji dotyczących dostępu do pamięci. Operacje postaci \texttt{Ustaw A:=B+C}, gdzie A, B i C są zmiennymi wspólnymi będziemy interpretować jako ciąg instrukcji\\
% % \begin{algorithm}[H]
% % \SetKwFunction{Gread}{global read}
% % \SetKwFunction{Gwrite}{global write}
% % \Gread{B, x}\;
% % \Gread{C, y}\;
% % \texttt{Ustaw} z:= x + y\;
% % \Gwrite{z,A}\;
% % \end{algorithm}
% \end{uwaga}

% \begin{figure}[h]
% \centering
% \includegraphics[width=34em]{images/pram_sum}
% \caption{Algorytm sumowania ośmiu elementów w modelu PRAM z osmioma procesorami. Każdy wewnętrzny wierzchołek grafu reprezentuje operację sumowania.}
% \label{fig:pram_sum}
% \end{figure}

Rozważmy problem obliczenia iloczynu \(\mathbf{C}\) dwóch macierzy \(\mathbf{A}\), \(\mathbf{B}\in\mathbb{R}^{n\times n}\), gdzie \(n=2^r\), dla pewnego \(r\in\mathbb{Z},\, r>0\). Załóżmy, że dysponujemy \(n^3\) procesorami \(P_{i,j,l}\), \(1\leq i, j, l \leq n\) maszyny CREW PRAM. 

\alglanguage{pseudocode}
\begin{algorithm}[h]
\centering
\begin{algorithmic}[1]
\Input{Macierze \(A\) i \(B\) umieszczone w pamięci wspólnej modelu CREW PRAM o \(n^3\) procesorach; zmienne lokalne służące do przechowywania rozmiaru \(n\), gdzie (\(n=2^r\) dla pewnego \(0<r\in \mathbb{Z}\)); numer procesora w postaci zmiennych \(i\), \(j\) oraz \(k\).}
\Tmp{Macierz \(T\) wymiaru \(n\times n\times n\) umieszczona w pamięci wspólnej; zmienna lokalna \(l\)}
\Output{Iloczyn macierzy \(C=AB\) w pamięci współdzielonej.}
% \item[]
\ParFor{\(P_{i,j,k}, \: 1 \leq i, \, j, \, k \leq n\)}
\State \(T_{i,j,k} \gets A_{i,k}B_{k,j}\)\Comment{Obliczanie składowych iloczynów skalarnych}
\EndParFor
\For{\(l \gets 1, \log{n}\)} \Comment{Sumowanie składowych iloczynów skalarnych}
	\ParFor{\(P_{i,j,k}, \: 1 \leq i, \, j \leq n, \, 1 \leq k \leq n/2^l \)}
\algstore{pramn3}
\end{algorithmic}
\caption{Algorytm mnożenia macierzy dla \(n^3\) procesorów.\cite{Czech} (cz. I)}
\label{alg:pram_multiply_1}
\end{algorithm}

\begin{algorithm}[h]
\ContinuedFloat
\centering
\begin{algorithmic}[1]
\algrestore{pramn3}
	\State \(T_{i,j,k} \gets T_{i,j,2k-1}+T_{i,j,2k}\)
\EndParFor
\EndFor
\ParFor{\(P_{i,j,k}, \: 1 \leq i, \, j \leq n, \, k = 1 \)}
\State \(C_{i,j} \gets T_{i,j,1} \)
\EndParFor
\end{algorithmic}
\caption{Algorytm mnożenia macierzy dla \(n^3\) procesorów.\cite{Czech} (cz. II)}
\label{alg:pram_multiply_1}
\end{algorithm}

\begin{uwaga}
Algorytm \ref{alg:pram_multiply_1} wymaga równoległego odczytu ponieważ w trakcie wykonania kroku (3) procesory \(P_{i,l,k}\) mogą równocześnie odczytywać te same dane. Przykładowo procesory \(P_{i,1,l},P_{i,2,l},\dots,P_{i,n,l}\) w trakcie wykonywania kroku (3) wszystkie wymagają dostępu do elementu \(A_{il}\).
\end{uwaga}

Obliczenia w wierszach (2-4) i (10-12) wykonywane są w czasie \(\mathcal{O}(1)\), a sumowanie składowych iloczynów skalarnych w wierszach (5-9) -- w czasie \(\mathcal{O}(\log{n})\). Złożoność, przyspieszenie, koszt i efektywność algorytmu są następujące\cite{Czech}:

\begin{align*}
T_p(n)|_{p=n^3} = T(n) &= \mathcal{O}(\log{n}), \\
S(n) &= \mathcal{O}(\frac{n^3}{\log{n}}), \\
C(n) &= \mathcal{O}(n^3 \log{n}), \\
E(n) &= \mathcal{O}(\frac{n^3}{n^3 \log{n}})=\mathcal{O}(\frac{1}{\log{n}}.
\end{align*}

\noindent Zgodnie z def. \ref{def:cost-optimal} algorytm \emph{nie} jest optymalny względem kosztu. Zaprezentujemy poniżej algorytm \ref{alg:pram_multiply_2} będący modyfikacją powyższego algorytmu wykonaną na \(n\) procesorach. Jak pokażemy poźniej, jego koszt jest optymalny.


\alglanguage{pseudocode}
\begin{algorithm}[H]
\centering
\begin{algorithmic}[1]
\Input{Macierze \(A\) i \(B\) umieszczone w pamięci wspólnej modelu CREW PRAM o \(n^3\) procesorach; zmienne lokalne służące do przechowywania rozmiaru \(n\), gdzie (\(n=2^r\) dla pewnego \(0<r\in \mathbb{Z}\)); numer procesora w postaci zmiennych \(i\), \(j\) oraz \(k\).}
\Tmp{Macierz \(T\) wymiaru \(n\times n\times n\); zmienna lokalna \(l\).}
\Output{Iloczyn macierzy \(C=AB\) w pamięci współdzielonej.}
% \item[]
\ParFor{\(P_{i}, 1\leq i \leq n\)}
	\For{\(j\gets 1, n\)}
		\For{\(k\gets 1, n\)}
		\algstore{pramn}
\end{algorithmic}
\caption{Algorytm mnożenia macierzy dla \(n\) procesorów.\cite{Czech} (cz. I)}
\label{alg:pram_multiply_2}
\end{algorithm}

\alglanguage{pseudocode}
\begin{algorithm}[H]
\centering
\begin{algorithmic}[1]		
\algrestore{pramn}
			\State \(T_{i,j,k} \gets A_{i,k}B_{k,j}\)\Comment{Obliczanie składowych iloczynów skalarnych}
		\EndFor
	\EndFor
\EndParFor
\For{\(l \gets 1, \log{n}\)} \Comment{Sumowanie składowych iloczynów skalarnych}
	\ParFor{\(P_{i}, 1\leq i \leq n\)}
		\For{\(j\gets 1, n\)}
			\For{\(k\gets 1, n/2^l\)}
				\State \(T_{i,j,k} \gets T_{i,j,2k-1}+T_{i,j,2k}\)
			\EndFor
		\EndFor
	\EndParFor
\EndFor

\ParFor{\(P_{i}, 1\leq i \leq n\)}
	\For{\(j\gets 1, n\)}
		\State \(C_{i,j} \gets T_{i,j,1} \)
	\EndFor
\EndParFor
\end{algorithmic}
\caption{Algorytm mnożenia macierzy dla \(n\) procesorów.\cite{Czech} (cz. II)}
\label{alg:pram_multiply_2}
\end{algorithm}


W algorytmie \ref{alg:pram_multiply_2} między \(n\) procesorów zostało rozdzielone \(n^3\) iloczynów \(A_{ik}B_{kj}\) tak, że w wierszach (2-8) procesor \(P_i\) oblicza elementy składowe iloczynów skalarnych \(i-tego\) wiersza macierzy \(A\) oraz wszystkich kolumn macierzy \(B\).

Najpierw każdy procesor \(P_i,\:(1\leq i \leq n)\) pobiera \(n\) elementów \(i\)-tego wiersza macierzy \(A\) oraz \(n^2\) elementów macierzy \(B\) i zapisuje do pamięci wspólnej \(n^2\) iloczynów \(t[i,j,k]\). Instrukcja w wierszu \(12\) wykonuje się na każdym procesorze \(n^2/2^l\) razy, gdzie \(l\) przebiega od \(1\) do \(\log{n}\). Łącznie daje to \(2n^2(1-\frac{1}{2r}\) wykonań wiersza \(12\) na procesor, przy czym każde wykonanie wymaga dwukrotnego odczytania danych z pamięci wspólnej, wykonania sumowania skalarów i zapisu wyniku. W wierszach (18-22) każdy z procesorów wykonuje \(n\) odczytów i zapisów do pamięci współdzielonej. Wynika stąd, że złożoność czasowa algorytmu \ref{alg:pram_multiply_2} to \(\mathcal{O}(n^2)\) i odpowiednio koszt algorytmu wynosi \(\mathcal{O}(n^3)\).

 W myśl definicji \ref{def:communication_complexity} i na podstawie powyższych rozważań złożoność komunikacyjna będzie równa  \(n + n^2 + n^2 + 3(2n^2(1-\frac{1}{2r})) + 2n = \mathcal{O}(n^2)\).


% \label{alg:crew_pram_multiplication}
% \begin{algorithm}[H]
% \SetKwData{Left}{left}\SetKwData{This}{this}\SetKwData{Up}{up}
% \SetKwFunction{Union}{Union}\SetKwFunction{FindCompress}{FindCompress}
% \SetKwInOut{Input}{wejście}\SetKwInOut{Output}{wyjście}
% \Input{Macierze \(\mathbf{A}\), \(\mathbf{B}\in\mathbb{R}^{n\times n}\), gdzie \(n=2^k\), dla pewnego \(k\in\mathbf{N}\) przechowywanych we wspólnej pamięci. Lokalnie zainicjalizowane zmienne to \(n\) i trójka wskaźników \((i, j, l)\)}
% \Output{Iloczyn \(\mathbf{C=AB}\) w pamięci współdzielonej}
% \begin{enumerate}
%  \item \( \mathtt{Oblicz}\quad C'(i,j,l) = A(i,l)B(l,j) \)
%  \item for \(h = 1\) to \( \log{n}\) do\\
%  if \(l \leq n/2^k\) then \texttt{Ustaw} \(C'(i,j,l):=C'(i,j,2l-1)+C'(i,j,2l)\)
%  \item \(if(l=1)\) then \texttt{Ustaw} \(C(i,j):=C'(i,j,1)\)
% \end{enumerate}
% \caption{Algorytm mnożenia macierzy w modelu PRAM\label{alg:pram_pseudokod}}
% \end{algorithm}

% \subsection{Blokowa procedura \texttt{gaxpy}}
% Załóżmy, że \(A, B, C \in \mathbb{R}^{n\times n}\), gdzie \(B\) jest macierzą górnie trójkątną oraz obliczenie wielokrotnie nadpisuje dane
% \begin{align}
% D=C+AB
% \end{align}
% na komputerze z pamięcią współdzieloną przez \(p\) węzły. Załóżmy ponadto, że \(n=rkp\)

% \begin{align}
% [D_1, D_2, \dots, D_{kp}] = [C_1, C_2, \dots, C_{k*p}] + [A_1, A_2, \dots, A_{kp}] [B_1, B_2, \dots, B_{kp}]
% \end{align}
% gdzie każdy blok ma szerekość \(r=n/(kp)\). Jeśli
% \begin{align*}
% B_j=
% \begin{pmatrix}
% B_{1j}\\
% \vdots\\
% B_{jj}\\
% 0\\
% \vdots\\
% 0\\
% \end{pmatrix},
% \quad
% B_{ij}\in\mathbb{R}^{r*r}
% \end{align*}
% wówczas
% \begin{align}
% D_j = C_j + AB_j = C_j + \sum_{r=1}^{j}A_r B_{rj}.
% \end{align}

