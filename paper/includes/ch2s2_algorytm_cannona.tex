\subsubsection{Wprowadzenie}
Powiedzmy, że chcemy przeprowadzić obliczenie \(D = C + AB\) gdzie \(A, B, C\in\mathbb{R}^{n\times n}\) w dwuwymiarowym torusie o rozmiarach \(p_1 \times p_1\) oraz, że \(n=rp_1\). Macierze \(A=(A_{ij})\), \(B=(B_{ij})\), \(C=(C_{ij})\) możemy rozpatrywać jako macierze blokowe, gdzie \(A_{ij}\), \(B_{ij}\), \(C_{ij}\) są macierzami \(r\times r\). Przyjmijmy, że węzeł \(P_{ij}\) zawiera blok \(A_{ij}\), \(B_{ij}\) i \(C_{ij}\) oraz, że jego zadanie polega na nadpisywaniu macierzy \(C_{ij}\) poprzez
\begin{align}\label{eq:cannon_per_node}
D_{ij}=C_{ij} + \sum_{k=1}^{p_1} A_{ik}B_{kj}.
\end{align}

Zanim przejdziemy do przypadku ogólnego, pokażemy algorytm dla przypadku \(p_1 = 3\). Rozważmy sieć w topologii dwuwymiarowego torusa \(3\times 3\) (rys. \ref{fig:cannon_torus1}).

\begin{figure}[h]
\centering
\begin{tabular}{|C{1.5cm}|C{1.5cm}|C{1.5cm}|}
\hline
\(P_{11}\) & \(P_{12}\) & \(P_{13}\) \\
\hline
\(P_{21}\) & \(P_{22}\) & \(P_{23}\) \\
\hline
\(P_{31}\) & \(P_{32}\) & \(P_{33}\) \\
\hline
\end{tabular}
\caption{Węzły w dwuwymiarowym torusie \(3\times 3\)}
\label{fig:cannon_torus1}
\end{figure}

\noindent Weźmy pod uwagę wyłącznie aktywność na węźle \(P_{11}\). Wykonuje on obliczenie

\begin{align}\label{eq:cannon_torus1}
D_{11}=C_{11}+A_{11}B_{11} + A_{12}B_{21} + A_{13}B_{31}.
\end{align}

\noindent Załóżmy, że rozmieściliśmy sześć podmacierzy \(A\) i \(B\) tak jak na rysunku \ref{fig:cannon_torus2}.

\begin{figure}[h]
\centering
\begin{tabular}{|cc|cc|cc|}
\hline
\(A_{11}\) & \(B_{11}\) & \(A_{12}\) & \(-\) & \(A_{13}\) & \(-\) \\
\hline
\(-\) & \(B_{21}\) & \(-\) & \(-\) & \(-\) & \(-\) \\
\hline
\(-\) & \(B_{31}\) & \(-\) & \(-\) & \(-\) & \(-\) \\
\hline
\end{tabular}
\caption{Wstępne rozmieszczenie macierzy blokowych \(A_{ij}\) i \(B_{ij}\) koniecznych do wykonania algorytmu tylko na węźle \(P_{11}\). Miejsca oznaczone znakiem ,,\(-\)'' w ogólności są przeznaczone dla pozostałych danych. Rozmieszczenie danych w sieci odpowiada rysunkowi 
\ref{fig:cannon_torus1}}
\label{fig:cannon_torus2}
\end{figure}

\noindent Algorytm polega na przesuwaniu wierszy powstałych z bloków macierzy zapisanych w węzłach sieci. W każdym kroku na wybranym dla przykładu węźle \(P_{11}\) wykonujemy lokalne obliczenia prowadzące do otrzymania wartości wyrażenia \eqref{eq:cannon_torus1}. Kolejne kroki algorytmu przedstawione są na rysunku \ref{fig:cannon_torus3}.


\begin{figure}[h]
\centering
\begin{tabular}{lr}
\begin{tabular}{|cc|cc|cc|}
\hline
\(A_{12}\) & \(B_{21}\) & \(A_{13}\) & \(-\) & \(A_{11}\) & \(-\) \\
\hline
\(-\) & \(B_{31}\) & \(-\) & \(-\) & \(-\) & \(-\) \\
\hline
\(-\) & \(B_{11}\) & \(-\) & \(-\) & \(-\) & \(-\) \\
\hline
\end{tabular} &
\hspace{1cm}\(C_{loc}=C_{loc}+A_{12}B_{21}\)
\end{tabular}

\vspace{0.5cm}

\begin{tabular}{lr}
\begin{tabular}{|cc|cc|cc|}
\hline
\(A_{13}\) & \(B_{31}\) & \(A_{11}\) & \(-\) & \(A_{12}\) & \(-\) \\
\hline
\(-\) & \(B_{11}\) & \(-\) & \(-\) & \(-\) & \(-\) \\
\hline
\(-\) & \(B_{21}\) & \(-\) & \(-\) & \(-\) & \(-\) \\
\hline
\end{tabular} &
\hspace{1cm}\(C_{loc}=C_{loc}+A_{13}B_{31}\)
\end{tabular}

\vspace{0.5cm}

\begin{tabular}{lr}
\begin{tabular}{|cc|cc|cc|}
\hline
\(A_{11}\) & \(B_{11}\) & \(A_{12}\) & \(-\) & \(A_{13}\) & \(-\) \\
\hline
\(-\) & \(B_{21}\) & \(-\) & \(-\) & \(-\) & \(-\) \\
\hline
\(-\) & \(B_{31}\) & \(-\) & \(-\) & \(-\) & \(-\) \\
\hline
\end{tabular} &
\hspace{1cm}\(C_{loc}=C_{loc}+A_{11}B_{11}\)
\end{tabular}
\caption{Rozmieszczenie danych dla trzech kroków metody Cannona z uwzględnienem danych koniecznych do obliczeń tylko na węźle \(P_{11}\)}
\label{fig:cannon_torus3}
\end{figure}

\noindent Po wykonaniu trzech kroków węzeł \(P_{11}\) ma w pamięci lokalnej macierz \(D_{11}\).

Przepływ danych został zorganizowany w taki sposób, że bloki \(A_{ij}\) przesuwanę są w siatce z prawej na lewą, zaś bloki \(B_{ij}\) --- z dołu na górę. Widać, że węzeł \(P_{11}\) musi wykonywać algorytm \ref{alg:cannon_1}.

\alglanguage{pseudocode}
\begin{algorithm}[H]
\centering
\begin{algorithmic}[1]
% \Input{macierze \(a[1\dots n, 1\dots n]\) i \(b[1\dots n, 1 \dots n]\) umieszczone w pamięci wspólnej modelu CREW PRAM o \(n^3\) procesorach; zmienne lokalne służące do przechowywania rozmiaru \(n\), gdzie (\(n=2^r\) dla pewnego \(0<r\in \mathbb{Z}\)); numer procesora w postaci zmiennych \(i\), \(j\) oraz \(k\)}
% \Tmp{macierz \(t[1\dots n, 1\dots, n, 1\dots, n]\) umieszczona w pamięci wspólnej; zmienna lokalna \(l\)}
% \Output{iloczyn macierzy \(c=ab\) w pamięci współdzielonej}
% \item[]
\For{\(i \gets 1, 3\)}
\Send{\(A_{loc}\), lewo}
\Send{\(B_{loc}\), góra}
\Recv{\(A_{loc}\), prawo}
\Recv{\(B_{loc}\), dół}
\State \(C_{loc} = C_{loc} + A_{loc}B_{loc}\)
\EndFor
\end{algorithmic}
\caption{Algorytm Cannona dla dwuwymiarowego torusa \(3\times 3\).}
\label{alg:cannon_1}
\end{algorithm}

\noindent W przyjętym modelu obliczeń zadziała również algorytm \ref{alg:cannon_2}:

\alglanguage{pseudocode}
\begin{algorithm}[H]
\centering
\begin{algorithmic}[1]
% \Input{macierze \(a[1\dots n, 1\dots n]\) i \(b[1\dots n, 1 \dots n]\) umieszczone w pamięci wspólnej modelu CREW PRAM o \(n^3\) procesorach; zmienne lokalne służące do przechowywania rozmiaru \(n\), gdzie (\(n=2^r\) dla pewnego \(0<r\in \mathbb{Z}\)); numer procesora w postaci zmiennych \(i\), \(j\) oraz \(k\)}
% \Tmp{macierz \(t[1\dots n, 1\dots, n, 1\dots, n]\) umieszczona w pamięci wspólnej; zmienna lokalna \(l\)}
% \Output{iloczyn macierzy \(c=ab\) w pamięci współdzielonej}
% \item[]
\For{\(i \gets 1, 3\)}
\Send{\(A_{loc}\), lewo}
\Recv{\(A_{loc}\), prawo}
\Send{\(B_{loc}\), góra}
\Recv{\(B_{loc}\), dół}
\State \(C_{loc} = C_{loc} + A_{loc}B_{loc}\)
\EndFor
\end{algorithmic}
\caption{Algorytm Cannona dla dwuwymiarowego torusa \(3\times 3\).}
\label{alg:cannon_2}
\end{algorithm}

Wykonanie algorytm \ref{alg:cannon_2} trwa nieco dłużej ze względu zatrzymanie wykonania programu dopóki macierz \(A_{loc}\) nie zostanie wysłana\footnote{W interfejsie MPI obydwa algorytmy traktowane literalnie wywołują zazębienie (ang. \emph{deadlock}) ze względu na blokującą funkcję MPI\_Send; istnieje szereg metod nieblokujących pozwalających na implementację obydwu algorytmów.}.

Rozważymy teraz aktywność węzłów \(P_{12}\), \(P_{13}\), \(P_{21}\), \(P_{31}\). W dotychczasowych roważaniach odpowiednio pomagały one jedynie w przesuwaniu bloków \(A_{11}\), \(A_{12}\), \(A_{13}\) oraz \(B_{11}\), \(B_{21}\) i \(A_{31}\). Gdyby \(B_{32}\), \(B_{12}\), \(B_{22}\) przechodziły przez węzeł \(P_{12}\) w trakcie wykonywania algorytmu, wówczas moglibyśmy na węźle \(P_{12}\) otrzymać wartość wyrażenia:
\begin{align*}
D_{12} = C_{12} + A_{13}B_{32} + A_{11}B_{12} + A_{12}B_{22}.
\end{align*}

\noindent Rozumując podobnie widzimy, że węzeł \(P_{13}\) mógłby obliczać wyrażenie:
\begin{align*}
D_{13} = C_{13} + A_{11}B_{13} + A_{12}B_{23} + A_{13}B_{33}.
\end{align*}
\noindent o ile \(B_{13}\), \(B_{23}\) i \(B_{33}\) znajdowałyby się na węźle odpowiednio dla kroków \(t=1, 2, 3\). Uwzględniając obliczenia na węzłach \(P_{12}\) i \(P_{13}\) możemy zainicjalizować sieć jak na rys. \ref{fig:cannon_torus_init}

\begin{figure}[h]
\centering
\begin{tabular}{|cc|cc|cc|}
\hline
\(A_{11}\) & \(B_{11}\) & \(A_{12}\) & \(B_{22}\) & \(A_{13}\) & \(B_{33}\) \\
\hline
\(-\) & \(B_{21}\) & \(-\) & \(B_{32}\) & \(-\) & \(B_{13}\) \\
\hline
\(-\) & \(B_{31}\) & \(-\) & \(B_{12}\) & \(-\) & \(B_{23}\) \\
\hline
\end{tabular}
\caption{Wstępne rozmieszczenie macierzy blokowych \(A_{ij}\) i \(B_{ij}\) koniecznych do wykonania obliczeń na węzłach \(P_{11}\), \(P_{12}\) i \(P_{13}\).}
\label{fig:cannon_torus_init}
\end{figure}

Zastosowanie odpowiednich przesunięć w algorytmie Cannona ilustruje rys. \ref{fig:cannon_torus4}.

\begin{figure}[H]
\centering
\begin{tabular}{lr}
\begin{tabular}{|cc|cc|cc|}
\hline
\(A_{12}\) & \(B_{21}\) & \(A_{13}\) & \(B_{32}\) & \(A_{11}\) & \(B_{13}\) \\
\hline
\(-\) & \(B_{31}\) & \(-\) & \(B_{12}\) & \(-\) & \(B_{23}\) \\
\hline
\(-\) & \(B_{11}\) & \(-\) & \(B_{22}\) & \(-\) & \(B_{33} \) \\
\hline
\end{tabular} &
\hspace{1cm}\(t=1\)
\end{tabular}

\vspace{0.5cm}

\begin{tabular}{lr}
\begin{tabular}{|cc|cc|cc|}
\hline
\(A_{13}\) & \(B_{31}\) & \(A_{11}\) & \(B_{12}\) & \(A_{12}\) & \(B_{23}\) \\
\hline
\(-\) & \(B_{11}\) & \(-\) & \(B_{22}\) & \(-\) & \(B_{33}\) \\
\hline
\(-\) & \(B_{21}\) & \(-\) & \(B_{32}\) & \(-\) & \(B_{13} \) \\
\hline
\end{tabular} &
\hspace{1cm}\(t=2\)
\end{tabular}

\vspace{0.5cm}

\begin{tabular}{lr}
\begin{tabular}{|cc|cc|cc|}
\hline
\(A_{11}\) & \(B_{11}\) & \(A_{12}\) & \(B_{22}\) & \(A_{13}\) & \(B_{33}\) \\
\hline
\(-\) & \(B_{21}\) & \(-\) & \(B_{32}\) & \(-\) & \(B_{13}\) \\
\hline
\(-\) & \(B_{31}\) & \(-\) & \(B_{12}\) & \(-\) & \(B_{23} \) \\
\hline
\end{tabular} &
\hspace{1cm}\(t=3\)
\end{tabular}
\caption{Rozmieszczenie danych dla trzech kroków metody Cannona z uwzględnienem danych koniecznych do obliczeń na węzłach \(P_{11}\), \(P_{12}\) i \(P_{13}\).}
\label{fig:cannon_torus4}
\end{figure}

\noindent Widać, że jeśli macierz \(B\) jest wstępnie rozmieszczona z zastosowaniem jednego przesunięcia, to po zakończeniu obliczeń w węzłąch \(P_{11}\), \(P_{12}\), \(P_{13}\) otrzymujemy pierwszy wiersz macierzy \(C\).

Rozważmy teraz aktywność na wszystkich dziewięciu węzłach sieci. Powiedzmy, że rozmieszczamy dane w sieci tak jak przedstawiono na rys. \ref{fig:cannon_init_2}.

\begin{figure}[H]
\centering
\begin{tabular}{lr}
\begin{tabular}{|cc|cc|cc|}
\hline
\(A_{11}\) & \(B_{11}\) & \(A_{12}\) & \(B_{22}\) & \(A_{13}\) & \(B_{33}\) \\
\hline
\(A_{22}\) & \(B_{21}\) & \(A_{23}\) & \(B_{32}\) & \(A_{21}\) & \(B_{13}\) \\
\hline
\(A_{33}\) & \(B_{31}\) & \(A_{31}\) & \(B_{12}\) & \(A_{32}\) & \(B_{23} \) \\
\hline
\end{tabular} &
\hspace{1cm}
\end{tabular}
\caption{Wstępne rozmieszczenie macierzy blokowych \(A_{ij}\) i \(B_{ij}\) w dwuwymiarowym torusie \(3\times 3\).}
\label{fig:cannon_init_2}
\end{figure}

\noindent Jeśli wstępnie przesuniemy drugi wiersz o jedną kolumnę w lewo i trzeci wiersz -- o dwie, wówczas możemy przeprowadzić odpowiednie dodawania i mnożenia występujące w wyrażeniu \eqref{eq:cannon_per_node} dla każdego węzła sieci w każdym kroku algorytmu. 

\begin{figure}[H]
\centering
\begin{tabular}{lr}
\begin{tabular}{|cc|cc|cc|}
\hline
\(A_{12}\) & \(B_{21}\) & \(A_{13}\) & \(B_{32}\) & \(A_{11}\) & \(B_{13}\) \\
\hline
\(A_{23}\) & \(B_{31}\) & \(A_{23}\) & \(B_{12}\) & \(A_{22}\) & \(B_{23}\) \\
\hline
\(A_{31}\) & \(B_{11}\) & \(A_{32}\) & \(B_{22}\) & \(A_{33}\) & \(B_{33} \) \\
\hline
\end{tabular} &
\hspace{1cm}\(t=1\)
\end{tabular}
\vspace{0.5cm}

\begin{tabular}{lr}
\begin{tabular}{|cc|cc|cc|}
\hline
\(A_{13}\) & \(B_{31}\) & \(A_{11}\) & \(B_{12}\) & \(A_{12}\) & \(B_{23}\) \\
\hline
\(A_{21}\) & \(B_{11}\) & \(A_{22}\) & \(B_{22}\) & \(A_{23}\) & \(B_{33}\) \\
\hline
\(A_{32}\) & \(B_{21}\) & \(A_{33}\) & \(B_{32}\) & \(A_{31}\) & \(B_{13} \) \\
\hline
\end{tabular} &
\hspace{1cm}\(t=2\)
\end{tabular}
\vspace{0.5cm}

\begin{tabular}{lr}
\begin{tabular}{|cc|cc|cc|}
\hline
\(A_{11}\) & \(B_{11}\) & \(A_{12}\) & \(B_{22}\) & \(A_{13}\) & \(B_{33}\) \\
\hline
\(A_{22}\) & \(B_{21}\) & \(A_{23}\) & \(B_{32}\) & \(A_{21}\) & \(B_{13}\) \\
\hline
\(A_{33}\) & \(B_{31}\) & \(A_{31}\) & \(B_{12}\) & \(A_{32}\) & \(B_{23} \) \\
\hline
\end{tabular} &
\hspace{1cm}\(t=3\)
\end{tabular}
\caption{Rozmieszczenie danych dla trzech kroków metody Cannona w dwuwymiarowym torusie \(3\times 3\).}
\label{fig:cannon_last_one}
\end{figure}

Teraz jesteśmy gotowi do przedstawienia ogólnej wersji algorytmu Cannona. 

\subsubsection{Uogólnienie}
Założymy, że węzły \(P_{ij}\) mają w pamięci lokalnej macierze \(A_{ij}\), \(B_{ij}\) i \(C_{ij}\). Żeby wstępnie przesunąć bloki macierzy \(A\) zauważmy, że \(i\)-ty wiersz węzłów sieci powinien przesłać macierze \(A_{ij}\) w lewo o \(i-1\) pozycji\footnote{Numerację elementów zaczynamy od 1.}. Podobnie, \(j\)-tą kolumnę węzłów przesuwamy w górę o \(j-1\) pozycji. Powyższe rozważania prowadzą do sformułowania algorytmu \ref{alg:cannon_algorithm}.


\alglanguage{pseudocode}
\begin{algorithm}[H]
\centering
\begin{algorithmic}[1]
%\Input{}
\Tmp{
\(p_1\),
\((\mu, \lambda)\) (współrzędne węzła w sieci),
\( \mathit{góra}\),
\( \mathit{dół}\),
\( \mathit{lewo}\),
\(\mathit{prawo}\) (identyfikatory sąsiadujących węzłów),
\( \mathit{wiersz} = 1 + (mu-1)r : \mu r\),
\(\mathit{kolumna} = 1 + (\lambda -1)r:\lambda r\),
\(B_{loc}=B(wiersz, kolumna)\),
\(A_{loc}=A(wiersz, kolumna)\),
\(C_{loc}=B(wiersz, kolumna)\)}
% \Output{}
\item[]
\For{\(k \gets 1, \mu - 1\)} \Comment{Wstępne przesunięcie \(A_{\mu j}\) i \(B_{i \lambda}\).}
	\Send{\(A_{loc}\), lewo}
	\Recv{\(A_{loc}\), prawo}
\algstore{cannon}
\end{algorithmic}
\caption{Algorytm Cannona\cite{Golub} (cz. I)}
%\label{alg:cannon_algorithm}
\end{algorithm}

%\alglanguage{pseudocode}
\begin{algorithm}[H]
\ContinuedFloat
\centering
\begin{algorithmic}[1]
\algrestore{cannon}
\EndFor
\For{\(k \gets 1, \lambda - 1\)}
	\Send{\(B_{loc}\), góra}
	\Recv{\(B_{loc}\), dół}
\EndFor
\For{\(k \gets 1, p_1\)} \Comment{Obliczanie iloczynu macierzy \(A_{loc}B_{loc}\)}
	\State \(C_{loc}=C_{loc} + A_{loc}B_{loc}\)
	\Send{\(A_{loc}\), lewo}
	\Send{\(B_{loc}\), góra}
	\Recv{\(A_{loc}\), prawo}
	\Recv{\(B_{loc}\), dół}
\EndFor
\For{\(k \gets 1, \mu -1\)} \Comment{Doprowadzanie rozkładu podmacierzy}
	\Send{\(A_{loc}\), prawo} \Comment{do stanu początkowego}
	\Recv{\(A_{loc}\), lewo}
\EndFor
\For{\(k \gets 1, \lambda -1\)}
	\Send{\(B_{loc}\), dół}
	\Recv{\(B_{loc}\), góra}
\EndFor
\end{algorithmic}
\caption{Algorytm Cannona\cite{Golub} (cz. II)}
\label{alg:cannon_algorithm}
\end{algorithm}

\subsubsection{Analiza algorytmu}
Powiedzmy, że \(A, B\) i \(C\) są macierzami wymiaru \(n\times n\) oraz \(n\) jest wielokrotnością \(\sqrt{p}\). Wówczas każdy proces odpowiada za obliczenie części macierzy wynikowej \(C\) o wymiarach \((n/\sqrt{p}) \times (n/\sqrt{p})\).

W trakcie każdej interacji algorytmu każdy proces mnoży blok wymiaru \((n/\sqrt{p}) \times (n/\sqrt{p})\) macierzy \(A\) z blokiem \((n/\sqrt{p}) \times (n/\sqrt{p})\) macierzy \(B\). Wynik takiego mnożenia jest dodawany jako iloczyn częściowy do bloku wynikowego macierzy \(C\). Jeśli przez \(\chi\) oznaczymi czas potrzebny do wykonania jednej operacji dodawania i mnożenia w jednym węźle, wówczas złoność czasowa każdej iteracji wynosi:
\[
\chi (n/\sqrt{p})^3 = \chi n^3 / p^{3/2}
\]
\noindent Algorytm kończy się po \(p\) iteracjach. Stąd czas wykonania powinien wynosić:
\begin{align}\label{eq:timecomplexity_cannon}
T_p(n) = \sqrt{p} \chi n^3 / p^{3/2} = \chi n^3 / p
\end{align}
\noindent Jeśli \(p=n^2\), wówczas algorytm wykonuje się w czasie liniowym.

Przed każdą iteracją proces musi wysłać blok macierzy \(A\) i blok macierzy \(B\) do odpowiedniego procesu i odebrać blok macierzy \(A\) i \(B\). Oznaczmy przez \(1/\beta\) czas potrzebny do przesłanai jednego elementu macierzy. Czas potrzebny do wstepnej dystrybucji macierzy w sieci możemy oszacować przez:
\begin{align}\label{eq:communicationtime_cannon1}
2\left(\lambda + \frac{n^2}{p\beta}\right)
\end{align}
Poprzez \(\sqrt{p}\) iteracji każdy proces musi wysłać swoje bloki macierzy \(A\) i \(B\) oraz odebrać nowe bloki, które później pomnoży. Całkowity czas potrzebny do komunikacji między procesami w takcie wykonywania algorytmu możemy oszacować przez:
\begin{align}\label{eq:communicationtime_cannon2}
2\sqrt{p}\left(\lambda + \frac{n^2}{p\beta}\right)
\end{align}

\noindent Dodając do siebie wyrażenia \eqref{eq:timecomplexity_cannon}, \eqref{eq:communicationtime_cannon1}, \eqref{eq:communicationtime_cannon2} możemy oszacować całkowity czas wykonania algorytmu Cannona:
\[
\chi n^3 / p^{3/2} + 2(\sqrt{p}+1)\left(\lambda + \frac{n^2}{p\beta}\right)
\]


Operacje (16-23) algorytmu \ref{alg:cannon_algorithm} nie są istotne dla samego mnożenia macierzy. Ich zadaniem jest doprowadzenie rozkładu podmacierzy \(A_{ij}, B_{ij}\) w torusie do stanu początkowego. Operacja taka daje możliwość wykorzystanie początkowego rozkładu danych do dalszych operacji na macierzach \(A\) i \(B\).


Podsumowująć: złożoność, przyspieszenie, koszt i efektywność algorytmu \ref{alg:cannon_algorithm} są następujące\cite{Czech}:
\begin{align*}
T_p(n)|_{p=n^2} = T(n) &= \mathcal{O}(n), \\
S(n) &= \mathcal{O}(\frac{n^3}{n}) = \mathcal{O}(n^2), \\
C(n) &= \mathcal{O}(n^2n)=\Theta(n^3), \\
E(n) &= \mathcal{O}(\frac{n^3}{n^3})=\Theta(1).
\end{align*}
Zgodnie z def. \ref{def:cost-optimal} algorytm jest optymalny względem kosztu.


\clearpage

Przykład \ref{ex:cannon_1} wygenerowano za pomocą własnej implementacji algorytmu Cannona wchodzącego w skład programu \texttt{pmm}, który przybliżamy w części \ref{app:improvements}. 

\begin{przyklad}\label{ex:cannon_1}
Pokażmy iloczyn dwóch macierzy wymiaru \(6 \times 6\) metodą Cannona w torusie \(3\times 3\) procesorów. Niech:
\begin{align*}
A=
\begin{pmatrix}
        8 & 5 & 6 & 1 & 2 & 3 \\
        3 & 3 & 1 & 5 & 3 & 9 \\ 
        9 & 2 & 9 & 0 & 4 & 9 \\ 
        2 & 0 & 8 & 8 & 3 & 4 \\ 
        6 & 7 & 6 & 7 & 5 & 0 \\ 
        2 & 5 & 7 & 8 & 7 & 1 \\
\end{pmatrix} & \quad 
B=
\begin{pmatrix}
        5 & 9 & 2 & 6 & 8 & 8 \\
        1 & 6 & 0 & 1 & 6 & 7 \\ 
        2 & 2 & 4 & 9 & 6 & 1 \\ 
        6 & 8 & 5 & 4 & 4 & 5 \\ 
        7 & 2 & 3 & 1 & 0 & 9 \\ 
        1 & 8 & 0 & 6 & 6 & 8 \\
\end{pmatrix}.\\
\end{align*}


\noindent Poszczególne kroki algorytmu pokazane są na rysunku \ref{fig:cannon_example1}.


\begin{figure}[H]
\scriptsize
\centering
\begin{tabular}{l}
\normalsize{\(t=1\)} \\
\begin{tabular}{|C{4.5cm}|C{4.5cm}|C{4.5cm}|}
\hline
\(\begin{pmatrix}
        5 & 9 \\
        1 & 6 \\    
\end{pmatrix}\)
\(\begin{pmatrix}
        8 & 5 \\
        3 & 3 \\    
\end{pmatrix}\) 
\(\begin{pmatrix}
        67 & 52 \\
        26 & 23 \\    
\end{pmatrix}\) &
\(\begin{pmatrix}
        2 & 6 \\
        0 & 1 \\    
\end{pmatrix}\)
\(\begin{pmatrix}
        9 & 0 \\
        8 & 8 \\    
\end{pmatrix}\) 
\(\begin{pmatrix}
        66 & 48 \\
        8 & 8 \\    
\end{pmatrix}\) &
\(\begin{pmatrix}
        8 & 8 \\
        6 & 7 \\    
\end{pmatrix}\)
 \(\begin{pmatrix}
        5 & 0 \\
        7 & 1 \\    
\end{pmatrix}\)
\(\begin{pmatrix}
        96 & 8 \\
        79 & 7 \\    
\end{pmatrix}\) \\
\hline
\(\begin{pmatrix}
        4 & 9 \\
        5 & 4 \\    
\end{pmatrix}\)
\(\begin{pmatrix}
        9 & 2 \\
        2 & 0 \\    
\end{pmatrix}\) 
\(\begin{pmatrix}
        54 & 8 \\
        53 & 10 \\    
\end{pmatrix}\) &
\(\begin{pmatrix}
        6 & 1 \\
        4 & 5 \\    
\end{pmatrix}\)
\(\begin{pmatrix}
        6 & 9 \\
        7 & 8 \\    
\end{pmatrix}\) 
\(\begin{pmatrix}
        43 & 50 \\
        59 & 68 \\    
\end{pmatrix}\) &
\(\begin{pmatrix}
        2 & 2 \\
        6 & 8 \\    
\end{pmatrix}\)
\(\begin{pmatrix}
        2 & 3 \\
        3 & 9 \\    
\end{pmatrix}\) 
\(\begin{pmatrix}
        10 & 24 \\
        36 & 90 \\    
\end{pmatrix}\) \\
\hline
\(\begin{pmatrix}
        0 & 9 \\
        6 & 8 \\    
\end{pmatrix}\)
\(\begin{pmatrix}
        6 & 7 \\
        2 & 5 \\    
\end{pmatrix}\) 
\(\begin{pmatrix}
        18 & 45 \\
        52 & 82 \\    
\end{pmatrix}\) &
\(\begin{pmatrix}
        7 & 2 \\
        1 & 8 \\    
\end{pmatrix}\)
\(\begin{pmatrix}
        6 & 1 \\
        1 & 5 \\    
\end{pmatrix}\) 
\(\begin{pmatrix}
        44 & 17 \\
        14 & 41 \\    
\end{pmatrix}\) &
\(\begin{pmatrix}
        3 & 1 \\
        0 & 6 \\    
\end{pmatrix}\)
\(\begin{pmatrix}
        4 & 9 \\
        3 & 4 \\    
\end{pmatrix}\) 
\(\begin{pmatrix}
        15 & 31 \\
        18 & 24 \\    
\end{pmatrix}\) \\
\hline
\end{tabular} 
\end{tabular}
\vspace{0.5cm}









\begin{tabular}{l}
\normalsize{\(t=2\)} \\
\begin{tabular}{|C{4.5cm}|C{4.5cm}|C{4.5cm}|}
\hline
\(\begin{pmatrix}
        2 & 6 \\
        0 & 1 \\    
\end{pmatrix}\)
\(\begin{pmatrix}
        9 & 2 \\
        2 & 0 \\    
\end{pmatrix}\)
\(\begin{pmatrix}
        97 & 56 \\
        28 & 23 \\    
\end{pmatrix}\) &
\(\begin{pmatrix}
        8 & 8 \\
        6 & 7 \\    
\end{pmatrix}\)
\(\begin{pmatrix}
        6 & 7 \\
        7 & 8 \\    
\end{pmatrix}\)
\(\begin{pmatrix}
        170 & 168 \\
        93 & 106 \\    
\end{pmatrix}\) &
\(\begin{pmatrix}
        5 & 9 \\
        1 & 6 \\    
\end{pmatrix}\)
 \(\begin{pmatrix}
        2 & 3 \\
        3 & 9 \\    
\end{pmatrix}\)
\(\begin{pmatrix}
        133 & 104 \\
        99 & 64 \\    
\end{pmatrix}\) \\
\hline
\(\begin{pmatrix}
        6 & 1 \\
        4 & 5 \\    
\end{pmatrix}\)
\(\begin{pmatrix}
        6 & 7 \\
        2 & 5 \\    
\end{pmatrix}\)
\(\begin{pmatrix}
        92 & 55 \\
        87 & 63 \\    
\end{pmatrix}\) &
\(\begin{pmatrix}
        2 & 2 \\
        6 & 8 \\    
\end{pmatrix}\)
\(\begin{pmatrix}
        6 & 1 \\
        1 & 5 \\    
\end{pmatrix}\)
\(\begin{pmatrix}
        57 & 62 \\
        103 & 114 \\    
\end{pmatrix}\) &
\(\begin{pmatrix}
        4 & 9 \\
        5 & 4 \\    
\end{pmatrix}\)
\(\begin{pmatrix}
        4 & 9 \\
        3 & 4 \\    
\end{pmatrix}\)
\(\begin{pmatrix}
        53 & 96 \\
        68 & 151 \\    
\end{pmatrix}\) \\
\hline
\(\begin{pmatrix}
        7 & 2 \\
        1 & 8 \\    
\end{pmatrix}\)
\(\begin{pmatrix}
        8 & 5 \\
        3 & 3 \\    
\end{pmatrix}\)
\(\begin{pmatrix}
        80 & 86 \\
        84 & 111 \\    
\end{pmatrix}\) &
\(\begin{pmatrix}
        3 & 1 \\
        0 & 6 \\    
\end{pmatrix}\)
\(\begin{pmatrix}
        9 & 0 \\
        8 & 8 \\    
\end{pmatrix}\)
\(\begin{pmatrix}
        79 & 25 \\
        62 & 89 \\    
\end{pmatrix}\) &
\(\begin{pmatrix}
        0 & 9 \\
        6 & 8 \\    
\end{pmatrix}\)
\(\begin{pmatrix}
        5 & 0 \\
        7 & 1 \\    
\end{pmatrix}\)
\(\begin{pmatrix}
        78 & 40 \\
        104 & 32 \\    
\end{pmatrix}\) \\
\hline
\end{tabular} 
\end{tabular}
\vspace{0.5cm}








\begin{tabular}{l}
\normalsize{\(t=3\)} \\
\begin{tabular}{|C{4.5cm}|C{4.5cm}|C{4.5cm}|}
\hline
\(\begin{pmatrix}
        8 & 8 \\
        6 & 7 \\    
\end{pmatrix}\)
\(\begin{pmatrix}
        6 & 7 \\
        2 & 5 \\    
\end{pmatrix}\)
\(\begin{pmatrix}
       161 & 152 \\
        78 & 100 \\    
\end{pmatrix}\) &
\(\begin{pmatrix}
        5 & 9 \\
        1 & 6 \\    
\end{pmatrix}\)
\(\begin{pmatrix}
        6 & 1 \\
        1 & 5 \\    
\end{pmatrix}\)
\(\begin{pmatrix}
        209 & 218 \\
        105 & 137 \\    
\end{pmatrix}\) &
\(\begin{pmatrix}
        2 & 6 \\
        0 & 1 \\    
\end{pmatrix}\)
 \(\begin{pmatrix}
        4 & 9 \\
        3 & 4 \\    
\end{pmatrix}\)
\(\begin{pmatrix}
        159 & 146 \\
        102 & 68 \\    
\end{pmatrix}\) \\
\hline
\(\begin{pmatrix}
        2 & 2 \\
        6 & 8 \\    
\end{pmatrix}\)
\(\begin{pmatrix}
        8 & 5 \\
        3 & 3 \\    
\end{pmatrix}\)
\(\begin{pmatrix}
        114 & 71 \\
        159 & 117 \\    
\end{pmatrix}\) &
\(\begin{pmatrix}
        4 & 9 \\
        5 & 4 \\    
\end{pmatrix}\)
\(\begin{pmatrix}
        9 & 0 \\
        8 & 8 \\    
\end{pmatrix}\)
\(\begin{pmatrix}
        165 & 134 \\
        180 & 146 \\    
\end{pmatrix}\) &
\(\begin{pmatrix}
        6 & 1 \\
        4 & 5 \\    
\end{pmatrix}\)
\(\begin{pmatrix}
        5 & 0 \\
        7 & 1 \\    
\end{pmatrix}\)
\(\begin{pmatrix}
        90 & 97 \\
        123 & 156 \\    
\end{pmatrix}\) \\
\hline
\(\begin{pmatrix}
        3 & 1 \\
        0 & 6 \\    
\end{pmatrix}\)
\(\begin{pmatrix}
        9 & 2 \\
        2 & 0 \\    
\end{pmatrix}\)
\(\begin{pmatrix}
        109 & 92 \\
        96 & 111 \\    
\end{pmatrix}\) &
\(\begin{pmatrix}
        0 & 9 \\
        6 & 8 \\    
\end{pmatrix}\)
\(\begin{pmatrix}
        6 & 7 \\
        7 & 8 \\    
\end{pmatrix}\)
\(\begin{pmatrix}
        142 & 97 \\
        154 & 195 \\    
\end{pmatrix}\) &
\(\begin{pmatrix}
        7 & 2 \\
        1 & 8 \\    
\end{pmatrix}\)
\(\begin{pmatrix}
        2 & 3 \\
        3 & 9 \\    
\end{pmatrix}\)
\(\begin{pmatrix}
        98 & 79 \\
        130 & 107 \\    
\end{pmatrix}\) \\
\hline
\end{tabular} 
\end{tabular}
\caption{Rozmieszczenie danych dla trzech kroków metody Cannona w dwuwymiarowym torusie \(3\times 3\). W każdej komórce pokazano zawartość zmiennych lokalnych odpowiednio \(A_{loc}\), \(B_{loc}\) i \(C_{loc}\).}
\label{fig:cannon_example1}
\end{figure}

\end{przyklad}