Przyspieszenia uzyskiwane ze wzorów \eqref{eq:amdahl} i \eqref{eq:gustafson_barsis} nie uwzględniają złożoności \(T_{p}^{O}(n)\) związanej z prowadzeniem obliczeń i dlatego ich wartości są większe niż uzyskiwane doświadczalnie.

Zgodnie z \ref{eq:supSpn} czas wykonywania algorytmu równoległego jest równy:
\begin{align}\label{eq:karp_flatt_1}
T_{p}(n)=T_{1}^{s}(n)+T_{p}^{0}(n) + \frac{T_{1}^{r}(n)}{p}
\end{align}

Jeśli przez \(f\) oznaczymy częśc operacji algorytmu, których nie można zrównoleglić (część \emph{inherentnie sekwencyjną}) oraz złożoność dodatkową wynikającą z organizacji obliczeń, to mamy:

\begin{align}\label{eq:karp_flatt_2}
f=\frac{T_{1}^{s}(n)+T_{p}^{O}(n)}{T_{1}(n)}
\end{align}

Z \eqref{eq:karp_flatt_1} i \eqref{eq:karp_flatt_2}:

\begin{align}
T_{p}(n)=T_{1}^{s}(n)+T_{p}^{O}(n)+\frac{T_{1}^{r}(n)}{p} = fT_{1}(n)+\frac{(1-f)T_{1}(n)}{p}
\end{align}

Dzieląć obie strony równania przez \(T_{1}(n)\) otrzymujemy

\begin{align}\label{eq:karp_flatt_3}
f=\frac{\frac{1}{S_{p}(n)}-\frac{1}{p}}{1-\frac{1}{p}}
\end{align}

Wyrażenie \eqref{eq:karp_flatt_3} nazywamy \textbf{miarą Karpa-Flatta}.

\begin{definicja}[Miara Karpa-Flatta]
Dla danego algorytmu równoległego \(R\) rozwiązującego problem \(P\) o rozmiarze \(n\) przy pomocy \(p>1\) procesorów, doświadczalnie wyznaczona część sekwencyjna obliczeń \(e\) można wyraża się przez
\begin{align}
e = \frac{1/\Psi - 1/p}{1 - 1/p}
\end{align}
\end{definicja}