Niech \(p\) oznacza liczbę procesorów, \(\sigma\) -- część czasu obliczeń algorytmu równoległego przypadającą na wykonanie obliczeń w sposób sekwencyjny, a \(\rho\) -- część czasu obliczeń algorytmu równoległego przypadającą na wykonywanie obliczeń w sposób równoległy takie, że \(\sigma+\rho=1\). Czas wykonania tego samego algorytmu w hipotetycznym komputerze sekwencyjnym jest proporcjonalny do sumy \(\sigma + p\rho\), gdzie wyrażenie \(p\rho\) odpowiada czasowi wykonania części równoległej obliczeń przez jeden procesor. Przyspieszenie, które zostałoby uzyskane, gdyby obliczenia równoległe zostały przeprowadzone w komputerze sekwencyjnym wyraża się przez:
\begin{equation}\label{eq:gustafson&barsis}
\Psi_{p}(n)\leq\frac{\sigma+p\rho}{\sigma+\rho}=\sigma+p\rho=\sigma+p\left(1-\sigma\right)=p+\left(1-p\right)\sigma
\end{equation}

Wzór \eqref{eq:gustafson&barsis} jest znany jako \textbf{prawo Gustafsona i	Barsisa}. 

\begin{definicja}[Prawo Gustafsona i Barsisa]
Dla danego algorytmu \(R\) rozwiązującego problem \(P\) ustalonego rozmiaru \(n\) na \(p\) procesorach oznaczmy przez \(\sigma\) część całkowitego czasu wykonania algorytmu. Wówczas maksymalne przyspieszenie \(\Psi\) algorytmu \(R\) spełnia nierówność:
\begin{align*}
\Psi_{p}(n) \leq p + (1-p)\sigma
\end{align*}
\end{definicja}

% Prawo Gustafsona i Barsisa określa tzw. \textbf{skalowane przyspieszenie}, ponieważ wraz ze zmianą liczby procesorów skaluje się odpowiednio rozmiar problemu, tak aby utrzymać stały czas obliczeń równoległych \cite{Czech}.