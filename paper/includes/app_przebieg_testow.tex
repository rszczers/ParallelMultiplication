Wyniki przedstawione w pracy można odtworzyć za pomocą zadań obliczeniowych systemu kolejkowego Torque zdefiniowanych w plikach \texttt{job.sh} i \texttt{job\_seq.sh} (patrz listing \ref{l:torque_jobs}). Poniżej pokażemy przykładową sesję pracy z programem. Zaczniemy od procesu kompilacji i skończymy na generowaniu wykresów.\\

Aktualną wersję programu pobieramy z repozytorium i kompilujemy źródła:
\begin{minted}{bash}
  $ git clone git@github.com:rszczers/ParallelMultiplication.git
  $ make all
\end{minted}

\noindent W katalogu \texttt{./build/} powstały pliki \texttt{pmm} i \texttt{gen}. Pierwszy z nich implementuje wybrane algorytmy i przeprowadza ich testy, drugi służy do generowania przykładowych macierzy na potrzeby testów.

Aby dodać zdefiniowane zadania obliczeniowe do systemu kolejkowania wykonujemy komendy:
\begin{minted}{bash}
  $ qsub job.sh
  $ qsub job_seq.sh
\end{minted}

Aby ocenić status wykonania naszego zadania używamy komendy \mintinline{bash}{qstat}. Szacowany czas wykonania możemy odczytać dzięki wywołaniu komendy\\ \mintinline{bash}{showstart id}, gdzie \texttt{id} to identyfikator zadania nadany przez system kolejkowy.


W katalogu \texttt{./debug/} po wykonaniu znajdują się wszystkie mierzone parametry wykonania algorytmów dla każdego wywołania zdefiniowanego w zadaniu. Przeanalizowane dane na potrzeby wykresów umieszczane są w katalogu \texttt{./gnuplot/data/}, gdzie ulegają dalszej obróbce. Wynik końcowy w postaci wykresów umieszczany jest w katalogu \texttt{./paper/includes/plots/} skąd są importowane do niniejszej pracy.

\begin{listing}[H]
\footnotesize
\inputminted{bash}{includes/listings/job.sh}
\caption{Plik \texttt{job.sh}}
\label{l:torque_jobs}
\end{listing}

\begin{listing}[H]
\footnotesize
\inputminted{bash}{includes/listings/job_seq.sh}
\caption{Plik \texttt{job\_seq.sh}}
\label{l:torque_jobs}
\end{listing}

% \begin{listing}
% \footnotesize
% \caption{Przykładowa sesja pracy z programem}
% \label{l:sample_session}
% \end{listing}


