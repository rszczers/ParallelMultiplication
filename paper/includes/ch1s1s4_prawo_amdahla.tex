W złożoności \(T_{p}(n)\) można wyróżnić operacje obliczeniowe, które muszą być wykonane sekwencyjnie, \(T^{s}_{1}(n)\), oraz obliczenia, które mogą być wykonane równolegle, \(T^{r}_{1}(n)\). Inaczej:
\begin{align}
T_{1}(n) = T^{s}_{1}(n) + T^{r}_{1}(n)
\end{align}
Zakładając, że obliczenia \(T^{r}(n)\) da się równomiernie rozdzielić między \(p\) procesorami, przyspieszenie \(S(p, n)\) wynosi wówczas
\begin{align}\label{eq:supSpn}
S(p, n) = \frac{T_{1}(n)}{T_{p}(n)}\leq\frac{T^{s}_{1}(n) + T^{r}_{1}(n)}{T^{s}_{1}(n) + T^{r}_{1}(n)/p + T^{o}_{p}(n)}
\end{align}
gdzie \(T^{o}_{p}(n)\) jest złożonością dodatkową wynikającą z organizacji obliczeń równoległych.


Rozważmy teraz algorytm sekwencyjny o złożoności \(T_{1}(n)\) rozwiązujący zadany problem \(P\) o ustalonym rozmiarze \(n\). Niech \(s\) oznacza część operacji algorytmu, która musi być wykonana sekwencyjnie, zaś \(r\) część operacji, która może być wykonana równolegle. Oznaczmy: \(T^{s}(n) = sT_{1}(n)\), \(T^{r}(n)=rT_{1}(n)\), gdzie \(s+r=1\). 


Przyspieszenie algorytmu uzyskane po jego zrównolegleniu można wyznaczyć upraszczając wzór \eqref{eq:supSpn} przez pominięcie złożoności \(T^{o}_{p}(n)\). Mamy wówczas:

\begin{equation*}\label{eq:amdahl}
\begin{split}
S(p, n) &= \frac{T_{1}(n)}{T_{p}(n)}\leq\\
&\leq \frac{T^{s}_{1}(n) + T^{r}_{1}(n)}{T^{s}_{1}(n) + T^{r}_{1}(n)/p + T^{o}_{p}(n)}\leq\\
&\leq \frac{sT_{1}(n) + rT_{1}(n)}{sT_{1}(n) + rT_{1}(n)/p} = \frac{s+r}{s+r/p} = \frac{1}{s+r/p}= \\
&= \left(s+\frac{1-s}{p}\right)^{-1}
\end{split}
\end{equation*}
gdzie \(s\) – część obliczeń w algorytmie które muszą być wykonane sekwencyjnie; \(p\) – liczba procesorów.\\
Otrzymany wzór \eqref{eq:amdahl} nazywamy \textbf{prawem Amdahla}. 

\begin{definicja}[Prawo Amdahla\cite{Quinn}]\label{def:prawo_amdahla}
Niech \(s\) będzie częścią operacji w algorytmie \(R\), która musi być wykonana sekwencyjnie, taką że \(0\leq s \leq 1\). Wówczas maksymalne przyspieszenie \(\phi\) osiągalne przez komputer równoległy z \(p\) procesorami wykonujący algorytm \(R\) spełnia nierówność:
\begin{align*}
\phi \leq \frac{1}{s+(1-s)/p}
\end{align*}
\end{definicja}
Nierówność z definicji \ref{def:prawo_amdahla} służy do wyznaczania górnego ograniczenia przyspieszenia będącego funkcją wielkości \(s\) oraz liczby procesorów \(p\) przy ustalonym rozmiarze problemu \(n\).

