%\setlist[enumerate]{itemsep=-1mm}
%\DeclareGraphicsExtensions{.eps}
\DeclareUnicodeCharacter{00A0}{ }
\linespread{1.25}
\pagestyle{plain}

\theoremstyle{definition} 
\newtheorem{przyklad}{Przykład}[chapter]
\newtheorem{definicja}{Definicja}[chapter]
\theoremstyle{plain}
\newtheorem{twierdzenie}{Twierdzenie}[chapter]
\newtheorem{lemat}{Lemat}[chapter]
\theoremstyle{remark}
\newtheorem{uwaga}{Uwaga}[chapter]
\newtheorem{wniosek}{Wniosek}[chapter]

% \SetKwProg{ParFor}{parfor}{}{}

\newcommand\BlockIf[1]{\KwSty{Start If} \\ #1 \\ \KwSty{End If}}
\newcommand\BlockElseIf[1]{\KwSty{Start Else If} \\ #1 \\ \KwSty{End Else If}}
\newcommand\BlockElse[1]{\KwSty{Start Else} \\ #1 \\ \KwSty{End Else}}

% Podłoga, sufit
\DeclarePairedDelimiter\ceil{\lceil}{\rceil}
\DeclarePairedDelimiter\floor{\lfloor}{\rfloor}

% algorithm -> Algorytm
\floatname{algorithm}{Algorytm}

\newcommand*\Let[2]{\State #1 $\gets$ #2}
\algrenewcommand\alglinenumber[1]{
    {\sf\footnotesize\addfontfeatures{Colour=888888,Numbers=Monospaced}#1}}

% declaration of the new block
\algblock{ParFor}{EndParFor}
% customising the new block
\algnewcommand\algorithmicparfor{\textbf{parfor}}
\algnewcommand\algorithmicpardo{\textbf{do}}
\algnewcommand\algorithmicendparfor{\textbf{end\ parfor}}
\algrenewtext{ParFor}[1]{\algorithmicparfor\ #1\ \algorithmicpardo}
\algrenewtext{EndParFor}{\algorithmicendparfor}