W ramach pracy licencjackiej zaimplementowano w języku C prosty system na potrzeby testowania wybranych algorytmów obliczania iloczynu macierzy dowolnych rozmiarów. Program wykorzystuje interfejsy programowania równoległego MPI-1 i OpenMP oraz bibliotekę matematyczną Intel MKL. Dla ułatwienia program używa interfejsu argp do parsingu argumentów wejściowych oraz wyświetlania opcji \texttt{--help} i \texttt{--version} w stylu GNU.


Proces testowania został całkowicie zautomatyzowany ze względu przeprowadzenia dużej liczby testów dla różnych danych wejściowych i na potrzeby graficznej prezentacji wydajności obliczeń dla różnej ilości procesów i wątków. 


Przy domyślnych ustawieniach po wykonaniu każdego zadania w katalogu \texttt{./debug/} tworzony jest plik \texttt{debug\_X}, gdzie \texttt{X} to data wykonania zadania w formacie uniksowym. Zawiera on dane o czasie wykonania programu, jego części sekwencyjnej, rozmiarze danych, liczbie procesów i wątków. Istnieje również opcja zapisywania danych częściowych każdego z procesów na każdym etapie wykonywania algorytmu. Pozwala to prześledzić wszystkie etapy obliczeń.

Zestaw skryptów w językach Perl i Bash analizują pliki wynikowe z katalogu ./debug/ i przetwarzają je na pliki danych programu Gnuplot. Odpowiednie wykresy generują się po wykonaniu testów.


Program jest elastyczny. Pozwala użytkownikowi mnożyć dowolne dwie wybrane macierze zapisane w pliku w formacie wierszowym i zapisać wynik w wybranym przez użytkownika położeniu. 


W ramach pracy powstał mały program \texttt{gen} do generowania plików zawierających ciągi liczb pseudolosowych i buforowanego zapisu takich danych do pliku. Program powstał na potrzeby szybkiego generowania przykładowych macierzy. Pozwala określić ilość i zakres generowanych liczb.


Implementacja algorytmu Cannona, chociaż sam algorytm w oryginalnej wersji pracuje tylko na macierzach kwadratowych o rozmiarze wielokrotności szerokości lub długości siatki procesow, działa dla macierzy dowolnych rozmiarów. Program przed wykonaniem obliczeń skaluje macierze do wymiarów wymaganych przez wybrany algorytm. 


Całość zarządzania jest zestawem celów zdefiniowanych w pliku Makefile i realizowanych przez narzędzie Make.


Program rozwijany jest w serwisie github pod adresem \url{http://github.com/rszczers/ParallelMultiplication}.


\begin{minted}
\inputminted{c}{listings/help.sh}
\end{minted}