Można wyróżnić trzy zasadnicze powody obecnego zainteresowania obliczeniami równoległymi.
\begin{enumerate}
\item{Stały spadek kosztów sprzętu komputerowego,}
\item{Rozwój VLSI (\emph{Very-large-scale integration}) do poziomu umożliwiającego projektowanie układów scalonych zawierających miliony tranzystorów na pojedyńczym chipie,}
\item{Osiągnięcie fizycznych ograniczeń czasu cyklu procesora w architekturze von Neumanna (rys \ref{fig:neumann}).}
\end{enumerate}

Obliczenia równoległe, w świetle ograniczeń fizycznych procesorów jednordzeniowych, są odpowiedzią na potrzebę wykonywania szybszych obliczeń. Szybsze obliczenia pozwalają na obliczenia w większej skali i otrzymywanie szybszych rozwiązań.