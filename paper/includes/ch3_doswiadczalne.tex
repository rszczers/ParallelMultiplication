Dane pomiarowe uzyskano na klastrze obliczeniowym Solaris dzięki uprzejmości Instytutu Matematyki UMCS. Specyfikacja techniczna węzłów zawarta została w tabeli \ref{tab:specs}. W trakcie testów używano tylko jednego procesora przypadającego na węzeł. Do budowy programu zawierającego implementacje wybranych algorytmów użyto kompilatora \texttt{mpiicc} z bibliotekami \texttt{mkl} i \texttt{openmp}. Wszystkie rozważane w tym rozdziale przyspieszenia są obliczane względem naiwnego algorytmu sekwencyjnego o złożoności \(\mathcal{O}(n^3)\).


Wykresy \ref{pl:cannon_seq_diff}, \ref{pl:cannon_dgemm_diff}, \ref{pl:compar_dgemm}, \ref{pl:cannon_omp} prezentują przyspieszenia wariantów algorytmu Cannona dla \(p=2^k,\,k=2,4,6,8\) procesów uzyskane przy użyciu 22 węzłów i macierzy wymiarów \((2048\times 2048)\), \((4096\times 4096)\), \((8192\times 8192)\).


Wykres \ref{pl:mono} przedstawia przyspieszenia uzyskane w zależności od liczby wątków OpenMP dla jednego procesu MPI wykonanego na 22 fizycznych węzłąch sieci dla procedury \texttt{cblas\_dgemm} z biblioteki MKL oraz dyrektywy \texttt{parallel for} interfejsu OpenMP.



Wykresy \ref{pl:scal_cannon_dgemm} i \ref{pl:scal_cannon_omp} przedstawiają przyspieszenia uzyskane dla obliczeń w topologi torusa \((2\times 2)\), \((3\times 3)\), \((4\times 4)\), \(\dots\), \((16\times 16)\). Ze względu na specyfikę algorytmu Cannona macierze wejściowe są wstępnie skalowane\footnote{Ich rozmiar jest powiększany do wielokrotności \(\sqrt{p}\) (gdzie \(p\) to liczba procesów) przez uzupełnienie macierzy zerami.} przed rozesłaniem do węzłów sieci. Zmienia to początkowy rozmiar poblemu \(n\) do \(\left( \floor{ \sqrt{\frac{n}{p}} } \sqrt{p} + \sqrt{n}\mod{\sqrt{p}}\right)^2\).

\begin{table}[H]
\begin{tabular}{|l|l|}
\hline
Procesor & 2x Intel Xeon X5650 (2,66 GHz, 6 rdzeni, 12 wątków) \\
\hline
Pamięć lokalna & 48GB DDR3 \\
\hline
System operacyjny & Debian 3.2.39-2 (Linux 3.2.0-4-amd64 x86\_64 kernel)\\
\hline
\end{tabular}
\caption{Specyfikacja techniczna węzłów klastra Solaris}
\label{tab:specs}
\end{table}

% \section{Warianty algorytmu Cannona}

\begin{figure}[H]
\centering
\footnotesize
\subimport{plots/}{cannon_diff_n.tex}
\caption{Przyspieszenie algorytmu Cannona na systemie równoległym składającym się z 22 fizycznych węzłów sieci. Wykres uwzględnia testy przeprowadzone na trzech zestawach danych rozmiaru \((2048\times 2048)\), \((4096\times 4096)\), \((8192\times 8192)\).}
\label{pl:cannon_seq_diff}
\end{figure}


\begin{figure}[H]
\centering
\footnotesize
\subimport{plots/}{cannon_mkl_diff_n.tex}
\caption{Przyspieszenie względne hybrydowego algorytmu Cannona z procedurą \texttt{cblas\_dgemm} uzyskane na systemie równoległym składającym się z 22 fizycznych węzłów sieci dla trzech zestawów danych różnego rozmiaru.}
\label{pl:cannon_dgemm_diff}
\end{figure}

\begin{figure}[H]
\centering
\footnotesize
\subimport{plots/}{cannon_dgemm.tex}
\caption{Zestawienie porównawcze przyspieszeń z wykresów \ref{pl:cannon_seq_diff} i \ref{pl:cannon_dgemm_diff}.} 
\label{pl:compar_dgemm}
\end{figure}

\begin{figure}[H]
\centering
\footnotesize
\subimport{plots/}{cannon_omp.tex}
\caption{Porównanie wydajności klasycznego algorytmu Cannona oraz wersji hybrydowej z dyrektywą \texttt{parallel for} dla różnej liczby wątków w systemie równoległym składającym się z 22 fizycznych węzłów.}
\label{pl:cannon_omp}
\end{figure}




% \section{Obliczenia z użyciem interfejsu OpenMP}

\begin{figure}[H]
\centering
\footnotesize
\subimport{plots/}{mono.tex}
\caption{Porównanie wydajności procedury \texttt{cblas\_dgemm} biblioteki MKL do mnożenia z pomocą dyrektywy \texttt{parallel for} interfejsu OpenMP. Wyniki uzyskano na systemie równoległym składającym się z 22 fizycznych węzłów.}
\label{pl:mono}
\end{figure}

% \section{Obliczenia ze skalowaniem macierzy}

\begin{figure}[H]
\centering
\footnotesize
\subimport{plots/}{scal_cannon_dgemm.tex}
\caption{Przyspieszenie hybrydowego algorytmu Cannona z operacją \texttt{cblas\_dgemm} (na wykresie Cannon-DGEMM) dla macierzy \(4096\times 4096\) wykonany na systemie równoległym złożonym z 8 węzłów.}
\label{pl:scal_cannon_dgemm}
\end{figure}

\begin{figure}[H]
\centering
\footnotesize
\subimport{plots/}{scal_cannon_omp.tex}
\caption{Przyspieszenie hybrydowego algorytmu Cannona z wykonaniem mnożenia z dyrektywą \texttt{parallel for} interfejsu OpenMP na poziomie procesu (na wykresie Cannon-OMP) dla macierzy \(4096\times 4096\) wykonany na systemie równoległym złożonym z 8 węzłów. Rozmiar problemu ze wzlędu na wstępne skalowanie macierzy jest zmienny i wynosi \(\left( \floor{ \sqrt{\frac{n}{p}} } \sqrt{p} + \sqrt{n}\mod{\sqrt{p}}\right)^2\). }
\label{pl:scal_cannon_omp}
\end{figure}

\subsubsection{Wnioski}
Przy wykonywaniu obliczenia iloczynu macierzy \((2048\times 2048)\) klasyczną metodą Cannona (rys. \ref{pl:cannon_seq_diff}) z uruchomionymi 64 procesami na 22 węzłach osiągamy optymalne (aczkolwiek stosunkowo niskie) dla tej implementacji przyspieszenie. Wiedząc, że każdy z 22 użytych węzłów dysponuje procesorem sześciordzeniowym widzimy, że dla liczby procesów \(p>132\) obliczenia nakładają się na siebie i nie wykonują równolegle. Stąd obserwujemy znaczny spadek przyspieszenia dla liczby procesów \(p=256\) (rys. \ref{pl:cannon_seq_diff}). Dla macierzy większych wymiarów widzimy, że wykonywanie współbieżne z liczbą procesów \(p>132\), w którym duża część procesów nie jest wykonywana równolegle, ma korzystny wpływ na przyspieszenie.


Lepiej pod względem przyspieszenia zachowują się algorytmy przedstawione na wykresach \ref{pl:cannon_dgemm_diff}, \ref{pl:compar_dgemm}. Widzimy, że dla obliczeń macierzy \((8192\times8192)\) najoptymalniejsze jest przeprowadzenie ich w konfiguracji po dwa wątki na cztery procesy. Wyraźnie widoczny jest też spadek przyspieszenia przy przeprowadzaniu obliczeń na zbyt dużej liczbie procesów i wątków.


Na wykresach \ref{pl:cannon_omp} widzimy, że dla metody Cannona z obliczeniami z dyrektywą \texttt{parallel for} na poziomie procesu, wydajność obliczeń jest najwyższa przy konfiguracji czterech wątków na cztery procesy. W pozostałych przypadkach (pomijąc te, w których ponosimy koszty nakładania się obliczeń przez uruchomienie zbyt wielkiej liczby procesów lub wątków) algorytm jest nieznacznie szybszy od klasycznej metody Cannona.


Z wykresu \ref{pl:mono} widzimy, że dla 22 wątków procedura oparta o dyrektywę \texttt{parallel for} osiąga maksymalne przyspieszenie. Procedura \texttt{cblas\_dgemm} od dziesięciu wątków w górę osiąga stosunkowo stałe przyspieszenie.


Najszybszą metodą, zgodnie z oczekiwaniami, okazało się wykonanie obliczeń procedurą \texttt{cblas\_dgemm} z biblioteki Intel MKL przy wykorzystaniu wielowątkowości biblioteki OpenMP. Osiągała ona przyspieszenia dwukrotnie większe niż hybrydowy algorytm Cannona korzystający z tej procedury w obliczeniach na poziomie procesu. Uzyskany słaby wynik metody Cannona należy tłumaczyć tutaj przeznaczeniem tej metody do obliczeń na dużych zbiorach danych (obliczeń typu \emph{big data}).