Niech \(p\) oznacza liczbę procesorów, \(\sigma\) -- część czasu obliczeń algorytmu równoległego przypadającą na wykonanie obliczeń w sposób sekwencyjny, a \(\rho\) -- część czasu obliczeń algorytmu równoległego przypadającą na wykonywanie obliczeń w sposób równoległy takie, że \(\sigma+\rho=1\). Czas wykonania tego samego algorytmu w hipotetycznym komputerze sekwencyjnym jest proporcjonalny do sumy \(\sigma + p\rho\), gdzie wyrażenie \(p\rho\) odpowiada czasowi wykonania części równoległej obliczeń przez jeden procesor. Przyspieszenie, które zostałoby uzyskane, gdyby obliczenia równoległe zostały przeprowadzone w komputerze sekwencyjnym wyraża się przez:
\begin{equation}\label{eq:gustafson_barsis}
\Psi_{p}(n)\leq\frac{\sigma+p\rho}{\sigma+\rho}=\sigma+p\rho=\sigma+p\left(1-\sigma\right)=p+\left(1-p\right)\sigma
\end{equation}

Wzór \eqref{eq:gustafson_barsis} jest znany jako \textbf{prawo Gustafsona i	Barsisa}. 

\begin{definicja}[Prawo Gustafsona i Barsisa\cite{Quinn}]
Dla danego algorytmu \(R\) rozwiązującego problem \(P\) ustalonego rozmiaru \(n\) na \(p\) procesorach oznaczmy przez \(\sigma\) część całkowitego czasu wykonania algorytmu. Wówczas maksymalne przyspieszenie \(\Psi\) algorytmu \(R\) spełnia nierówność:
\begin{align*}
\Psi_{p}(n) \leq p + (1-p)\sigma
\end{align*}
\end{definicja}
\begin{uwaga}
Prawo Gustafsona i Barsisa określa tzw. \textbf{skalowane przyspieszenie}, ponieważ wraz ze zmianą liczby procesorów skaluje się odpowiednio rozmiar problemu, tak aby utrzymać stały czas obliczeń równoległych (z założenia \(\sigma + \rho = 1\)\cite{Czech}.
\end{uwaga}

\begin{przyklad}
Powiedzmy, że pewien program wykonany na 64 procesorach wykonuje się w czasie 220 sekund. Testy pokazują, że 5\% czasu wykonania przeznaczone jest na obliczenia sekwencyjne. Skalowane przyspieszenie tego programu wynosi:

\begin{align*}
\Psi_{64}(n) \leq 64 + (1-64)(0.05) = 64 - 3,15 = 60,85.
\end{align*}
\end{przyklad}