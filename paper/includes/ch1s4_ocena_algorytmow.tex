\subsection{Złożonośc czasowa}
\label{subsec:algorytmy_sekwencyjne}
\subsubsection{Algorytmy sekwencyjne}

Ograniczenia zasobów (np. czasu i przestrzeni) wymagane przez algorytmy sekwencyjne mierzymy jako funkcję rozmiaru danych wejściowych \(T(n)\), tzw. złożoność czasową. Ograniczenia te wyrażamy asymptotycznie używając notacji:

\begin{enumerate}
\item{\(T(n) = O(f(n))\), jeśli istnieje dodatnie stałe \(c\) i \(n_0\) takie, że \(\forall{n \geq n_0}: (T(n)\leq cf(n)) \)}
\item{\(T(n) = \Omega(f(n))\), jeśli istnieje dodatnie stałe \(c\) i \(n_0\) takie, że \(\forall{n \geq n_0}: (T(n)\geq cf(n)) \)}
\item{\(T(n) = \Theta(f(n))\), jeśli \(T(n)=O(f(n))\) i \(T(n)=\Omega(f(n))\)}
\end{enumerate}
Czas działania algorytmu sekwencyjnego szacuje się przez liczbę operacji podstawowych wymaganych przez algorytm jako funkcję ilości danych wejściowych.
\subsubsection{Algorytmy równoległe}

\begin{definicja}[Pesymistyczna złożoność obliczeniowa\cite{Czech}]\label{def:pesymistyczna_zlozonosc_czasowa}
Załóżmy że algorytm równoległy \(R\) rozwiązuje problem \(P\) o rozmiarze \(n\). \textbf{Pesymityczną złożonością czasową algorytmu równoległego} \(R\) nazywamy funkcję:\\
\begin{align}
T_{p}(n) = \sup_{d\in{D_n}}{\left\{t(p,d)\right\}},
\end{align}
gdzie \(t(p,d)\) oznacza liczbę kroków obliczeniowych (operacji dominujących) wykonanych dla zestawu danych \(d\) od momentu rozpoczęcia obliczeń algorytmu \(R\) przez pierwszy procesor do chwili zakończenia obliczeń przez wszystkie procesory, \(p\) -- liczbę procesorów, \(D_n\) -- zbiór wszystkich zestawów danych wejściowych \(d\) o rozmiarze \(n\).
\end{definicja}

\subsection{Przyspieszenie}
Potencjalną korzyść z równoległego wykonania zadania obliczeniowego możemy zmierzyć licząć czas jaki zajmuje wykonanie go na jednym procesorze i porównanie wyniku z wykonaniem tego samego zadania równolegle na \(N\) procesorach. 

\begin{definicja}[Przyspieszenie bezwzględne\cite{Czech}]
Niech \(P\) będzie pewnym zadaniem obliczeniowym, \(n\) -- rozmiarem danych wejjściowych. Wówczas

\begin{align}\label{def:speedup_abs}
 S_{p}(n)=\frac{T^{*}(n)}{T_{p}(n)(N)}
\end{align}

gdzie \(T^{*}(n)\) jest pesymistyczną złożonością czasową najszybszego znanego algorytmu sekwencyjnego \(R_s\) rozwiązującego problem \(P\) na jednym procesorze, \(T_{n}(N)\) jest pesymistyczną złożonością algorytmu \(R\), gdzie \(R\) jest równoległą wersją algorytmu \(R_s\). Wyrażenie \ref{def:speedup_abs} nazywamy \textbf{przyspieszeniem bezwzględnym} algorytmu \(R\).
\end{definicja}

\begin{wniosek}
Zgodnie z definicją \ref{def:pesymistyczna_zlozonosc_czasowa} przez \(T_{1}(n)\) rozumiemy złożoność algorytmu równoległego \(R\) wykonywanego przy użyciu jednego procesora. Jeśli algorytm \(R\) nie jest najlepszą równoległą wersją znanego algorytmu sekwencyjnego, to równość \(T_{1}(n) = T^{*}(n)\) nie zachodzi.
\end{wniosek}

\begin{uwaga}
Maksymalną wartością przyspieszenia \(S(p,n)\) jest \(p\), ponieważ używając \(p\) procesorów można przyspieszyć obliczenia najlepszego algorytmu sekwencyjnego co najwyżej \(p\) razy. Zwykle uzyskiwane przyspieszenie jest mniejsze niż \(p\). Przyczyną tego może być niewystarczający stopień zrównoleglenia problemu \(P\), opóźnienia w komunikacji między procesami lub narzut czasu wykonania spowodowane synchronizacją procesów.
\end{uwaga}

\begin{uwaga}
Istnieją problemy dla których najlepszy znany algorytm sekwencyjny \(R_s\) nie może zostać zrównoleglony. Wówczas równoległe rozwiązanie problemu w postaci pewnego algorytmu \(R\) działa na innej zasadzie. Wówczas pomocne w ocenie korzyści z jest posługiwanie się \emph{przyspieszeniem względnym}.
\end{uwaga}

\begin{definicja}[Przyspieszenie względne\cite{Czech}]
Niech \(P\) będzie pewnym zadaniem obliczeniowym, \(n\) -- rozmiarem danych wejściowych. Wówczas
\begin{align}\label{def:speedup_rel}
 S_{p}(n)=\frac{T_{1}(n)}{T_{p}(n)}
\end{align}
gdzie \(T_{1}(n)\) jest pesymistyczną złożonością czasową algorytmu równoległego \(R\) rozwiązującego problem \(P\) na jednym procesorze, \(T_{n}(N)\) jest pesymistyczną złożonością algorytmu \(R\) wykonanego na \(n\) procesorach. Wyrażenie \ref{def:speedup_abs} nazywamy \textbf{przyspieszeniem względnym} algorytmu \(R\).
\end{definicja}

\subsection{Koszt}
\begin{definicja}[Koszt algorytmu\cite{Czech}]\label{def:cost}
Niech \(T_{p}(n)\) będzie pędzie pesymistyczną złożonością obliczeniową algorytmu \(R\) dla \(p\) procesorów. Wówczas funkcję
\begin{align}
C_{p}(n) = p T_{p}(n)
\end{align}
nazywamy kosztem algorytmu \(R\) dla \(p\) procesorów.
\end{definicja}

W myśl definicji \ref{def:pesymistyczna_zlozonosc_czasowa} koszt algorytmu możemy rozumieć przez analogię do liczby operacji dominujących wykonanych łącznie przez wszystkie procesory. 

\begin{wniosek}
Łatwo widać, że koszt osiąga minimalną wartość \(C_{1}(n) = T^{*}(n)\) dla  najlepszego znanego algorytmu sekwencyjnego. Stąd koszt algorytmu równoległego \(R\) jest minimalny wtedy i tylko wtedy, gdy wykonywane są w nim tylko te operacje, które są wykonywane w najlepszym algorytmie sekwencyjnym \(R_s\).
\end{wniosek}

\begin{uwaga}
W praktyce uzyskanie równości kosztów \(pT_{p}(n)=T^{*}(n)\) wymaga minimalizacji komunikacji między procesorami lub uruchomienia algorytmów na architekturach w których komunikacja odbywa się na tyle szybko, że jej dodatkowe koszty są pomijalne. Różnicę między kosztem wykonania algorytmu równoległego a kosztem wykonania najlepszego algorytmu sekwencyjnego nazywamy \emph{kosztem organizacji obliczeń równoległych}.
\end{uwaga}

\begin{definicja}[Koszt organizacji obliczeń]
Różnicę 
\begin{align}
C_{p}^{O}(n) = C_{p}(n) - T^{*}(n) = pT_{p}(n) - T^{*}(n)
\end{align}
nazywamy \emph{kosztem ogranizacji obliczeń równoległych} algorytmu \(R\) dla problemu \(P\) o rozmiarze \(n\)
\end{definicja}

\begin{definicja}[Koszt optymalny]\label{def:cost-optimal}
Mówimy, że koszt algorytmu \(R\) jest \emph{optymalny}, jeśli koszt obliczeń równoległych \(C_{p}(n)\) jest asymptotycznie równy minimalnemu kosztowi obliczeń sekwencyjnych \(T^{*}(n)\), czyli:
\begin{align}
C_{p}(n) = \Theta(T^{*}(n))
\end{align}
\end{definicja}

\subsection{Efektywność}
\begin{definicja}[Efektywność\cite{Czech}]
Niech \(T_{p}(n)\) będzie pesymistyczną złożonością czasową algorytmu \(R\) dla \(p\) procesorów i problemu \(R\) o rozmiarze \(n\). Wówczas mamy
\begin{equation}
\begin{split}
E_{p}(n)  = \frac{T_{1}(n)}{p T_{p}(n)}
	      = \frac{T_{1}(n)}{C_{p}(n)}
          = \frac{S_{p}(n)}{p}
\end{split}
\end{equation}

Funkcję \(E_{p}(n)\) nazywamy efektywnością wykorzystania procesorów algorytmu \(R\).
\end{definicja}



\subsection{Prawo Amdahla}
W złożoności \(T_{p}(n)\) można wyróżnić operacje obliczeniowe, które muszą być wykonane sekwencyjnie, \(T^{s}_{1}(n)\), oraz obliczenia, które mogą być wykonane równolegle, \(T^{r}_{1}(n)\). Inaczej:
\begin{align}
T_{1}(n) = T^{s}_{1}(n) + T^{r}_{1}(n)
\end{align}
Zakładając, że obliczenia \(T^{r}(n)\) da się równomiernie rozdzielić między \(p\) procesorami, przyspieszenie \(S(p, n)\) wynosi wówczas
\begin{align}\label{eq:supSpn}
S_{p}(n) = \frac{T_{1}(n)}{T_{p}(n)}\leq\frac{T^{s}_{1}(n) + T^{r}_{1}(n)}{T^{s}_{1}(n) + T^{r}_{1}(n)/p + T^{o}_{p}(n)}
\end{align}
gdzie \(T^{o}_{p}(n)\) jest złożonością dodatkową wynikającą z organizacji obliczeń równoległych.


Rozważmy teraz algorytm sekwencyjny o złożoności \(T_{1}(n)\) rozwiązujący zadany problem \(P\) o ustalonym rozmiarze \(n\). Niech \(s\) oznacza część operacji algorytmu, która musi być wykonana sekwencyjnie, zaś \(r\) część operacji, która może być wykonana równolegle. Oznaczmy: \(T^{s}(n) = sT_{1}(n)\), \(T^{r}(n)=rT_{1}(n)\), gdzie \(s+r=1\). 


Przyspieszenie algorytmu uzyskane po jego zrównolegleniu można wyznaczyć upraszczając wzór \eqref{eq:supSpn} przez pominięcie złożoności \(T^{o}_{p}(n)\). Mamy wówczas:

\begin{equation}
\begin{split}\label{eq:amdahl}
S_{p}(n) &= \frac{T_{1}(n)}{T_{p}(n)}\leq\\
&\leq \frac{T^{s}_{1}(n) + T^{r}_{1}(n)}{T^{s}_{1}(n) + T^{r}_{1}(n)/p + T^{o}_{p}(n)}\leq\\
&\leq \frac{sT_{1}(n) + rT_{1}(n)}{sT_{1}(n) + rT_{1}(n)/p} = \frac{s+r}{s+r/p} = \frac{1}{s+r/p}= \\
&= \left(s+\frac{1-s}{p}\right)^{-1}
\end{split}
\end{equation}
gdzie \(s\) – część obliczeń w algorytmie które muszą być wykonane sekwencyjnie; \(p\) – liczba procesorów.\\
Otrzymany wzór \eqref{eq:amdahl} nazywamy \textbf{prawem Amdahla}. 


\begin{definicja}[Prawo Amdahla\cite{Quinn}]\label{def:prawo_amdahla}
Niech \(s\) będzie częścią operacji w algorytmie \(R\), która musi być wykonana sekwencyjnie, taką że \(0\leq s \leq 1\). Wówczas maksymalne przyspieszenie \(\Psi\) osiągalne przez komputer równoległy z \(p\) procesorami wykonujący algorytm \(R\) spełnia nierówność:
\begin{align}
\Psi_{p}(n) \leq \frac{1}{s+(1-s)/p}
\end{align}
\end{definicja}
\begin{uwaga}
Nierówność z definicji \ref{def:prawo_amdahla} służy do wyznaczania górnego ograniczenia przyspieszenia będącego funkcją wielkości \(s\) oraz liczby procesorów \(p\) przy ustalonym rozmiarze problemu \(n\).
\end{uwaga}

\begin{przyklad}
Przypuśćmy, że dysponujemy sześciordzeniowym procesorem i chcemy ocenić czy warto szukać równoległej wersji programu dla rozwiązania pewnego problemu. Ustaliliśmy, że 90\% czasu wykonania programu przeznacza się na wykonanie pewnej funkcji, którą chcemy zrównoleglić. Pozostałe 10\% czasu wykonania zajmują funkcję, które musimy wykonywać na jednym procesorze. Chcemy ocenić największe przyspieszenie jakiego możemy się spodziewać po równoległej wersji naszego programu. W tym celu możemy skorzystać z prawa Amdahla. Mamy:
\begin{align*}
S_{p}(n) \leq \frac{1}{0.1 + (1-0.1)/6} = 4.
\end{align*}
\noindent Powinniśmy się zatem spodziewać przyspieszenia o wartości co najwyżej 4.
\end{przyklad}


\begin{wniosek}
Przechodząć z wyrażeniem \eqref{eq:amdahl} do granicy \(p\to\infty\) mamy \(\lim_{p\to\infty}\frac{1}{s+(1-s)/p} = \frac{1}{s}\). Widać, że maksymalne przyspieszenie \(S_{p}(n)\), jakie można osiągnać nie zależy od liczby użytch procesorów \(p\), ale od ilości obliczeń sekwencyjnych \(s\) (pomijając dodatkowe koszty ogranizacji obliczeń).
\end{wniosek}

\begin{przyklad}
Przypuśćmy, że 25\% operacji w algorytmie równoległym musi być wykonanych równolegle. Wówczas maksymalne osiągalne przyspieszenie przy przeprowadzaniu obliczeń na coraz większej liczbie procesorów wynosi:
\begin{align*}
\lim_{p\to \infty} \frac{1}{0.25+(1-0.25)/p} = 4.
\end{align*}
\end{przyklad}

\begin{przyklad}
Powiedzmy, że zaimplementowaliśmy równoległą wersję pewnego algorytmu sekwencyjnego o złożoności czasowej \(\Theta(n^2)\), gdzie \(n\) to rozmiar danych.
Sekwencyjna część programu potrzebuje do samego załadowania danych oraz wyświetlenia wyniku obliczeń czasu \((18000+n) \; \mu s\). Część obliczeniowa jest równoległa i potrzebuje czasu \((n^2/100)\;\mu s\). Dla danych o rozmiarze \(10\,000\) stosując prawo Amdahla możemy ocenić maksymalne przyspieszenie jakie uzyskamy rozwiązując problem równolegle:
\begin{align*}
S_{p}(n) \leq \frac{(28\,000 + 1\,000\,000) \mu s}{(28\,000 + 1\,000\,000/p) \mu s}
\end{align*}

\end{przyklad}


\subsection{Prawo Gustafsona i Barsisa}
Niech \(p\) oznacza liczbę procesorów, \(\sigma\) -- część czasu obliczeń algorytmu równoległego przypadającą na wykonanie obliczeń w sposób sekwencyjny, a \(\rho\) -- część czasu obliczeń algorytmu równoległego przypadającą na wykonywanie obliczeń w sposób równoległy takie, że \(\sigma+\rho=1\). Czas wykonania tego samego algorytmu w hipotetycznym komputerze sekwencyjnym jest proporcjonalny do sumy \(\sigma + p\rho\), gdzie wyrażenie \(p\rho\) odpowiada czasowi wykonania części równoległej obliczeń przez jeden procesor. Przyspieszenie, które zostałoby uzyskane, gdyby obliczenia równoległe zostały przeprowadzone w komputerze sekwencyjnym wyraża się przez:
\begin{equation}\label{eq:gustafson_barsis}
\Psi_{p}(n)\leq\frac{\sigma+p\rho}{\sigma+\rho}=\sigma+p\rho=\sigma+p\left(1-\sigma\right)=p+\left(1-p\right)\sigma
\end{equation}

Wzór \eqref{eq:gustafson_barsis} jest znany jako \textbf{prawo Gustafsona i	Barsisa}. 

\begin{definicja}[Prawo Gustafsona i Barsisa\cite{Quinn}]
Dla danego algorytmu \(R\) rozwiązującego problem \(P\) ustalonego rozmiaru \(n\) na \(p\) procesorach oznaczmy przez \(\sigma\) część całkowitego czasu wykonania algorytmu. Wówczas maksymalne przyspieszenie \(\Psi\) algorytmu \(R\) spełnia nierówność:
\begin{align*}
\Psi_{p}(n) \leq p + (1-p)\sigma
\end{align*}
\end{definicja}
\begin{uwaga}
Prawo Gustafsona i Barsisa określa tzw. \textbf{skalowane przyspieszenie}, ponieważ wraz ze zmianą liczby procesorów skaluje się odpowiednio rozmiar problemu, tak aby utrzymać stały czas obliczeń równoległych (z założenia \(\sigma + \rho = 1\)\cite{Czech}.
\end{uwaga}

\subsection{Miara Karpa-Flatta}
Przyspieszenia uzyskiwane ze wzorów \eqref{eq:amdahl} i \eqref{eq:gustafson_barsis} nie uwzględniają złożoności \(T_{p}^{O}(n)\) związanej z prowadzeniem obliczeń i dlatego ich wartości są większe niż uzyskiwane doświadczalnie.

Zgodnie z wyrażeniem \ref{eq:supSpn} czas wykonywania algorytmu równoległego jest równy:
\begin{align}\label{eq:karp_flatt_1}
T_{p}(n)=T_{1}^{s}(n)+T_{p}^{0}(n) + \frac{T_{1}^{r}(n)}{p}
\end{align}

Jeśli przez \(f\) oznaczymy częśc operacji algorytmu, których nie można zrównoleglić (część \emph{inherentnie sekwencyjną}) oraz złożoność dodatkową wynikającą z organizacji obliczeń, to mamy:

\begin{align}
f=\frac{T_{1}^{s}(n)+T_{p}^{O}(n)}{T_{1}(n)}\label{eq:karp_flatt2}
\end{align}

Z \eqref{eq:karp_flatt_1} i \eqref{eq:karp_flatt2}:

\begin{align}
T_{p}(n)=T_{1}^{s}(n)+T_{p}^{O}(n)+\frac{T_{1}^{r}(n)}{p} = fT_{1}(n)+\frac{(1-f)T_{1}(n)}{p}
\end{align}

Dzieląć obie strony równania przez \(T_{1}(n)\) otrzymujemy

\begin{align}\label{eq:karp_flatt_3}
f=\frac{\frac{1}{S_{p}(n)}-\frac{1}{p}}{1-\frac{1}{p}}
\end{align}

Wyrażenie \eqref{eq:karp_flatt_3} nazywamy \textbf{miarą Karpa-Flatta}.

\begin{definicja}[Miara Karpa-Flatta]
Dla danego algorytmu równoległego \(R\) rozwiązującego problem \(P\) o rozmiarze \(n\) przy pomocy \(p>1\) procesorów, doświadczalnie wyznaczona część sekwencyjna obliczeń \(f\) wyraża się przez
\begin{align}
f = \frac{1/S_{p}(n) - 1/p}{1 - 1/p}
\end{align}
\end{definicja}

\begin{przyklad}\label{ex:karp1}
Powiedzmy, że testując algorytm równoległy na \(1, 2, \dots, 8\) procesorach otrzymaliśmy pewne przyspieszenia w zależności od ilości procesów i z wyrażenia \ref{eq:karp_flatt_3} obliczyliśmy eksperymentalnie wyznaczoną część sekwencyjną f. Dane zebrane są w tablicy \ref{tab:karp_flat1}.

\begin{table}[H]
\centering
\begin{tabular}{|c|c|c|c|c|c|c|c|}
\hline
\(p\) & \(2\) & \(3\) & \(4\) & \(5\) & \(6\) & \(7\) & \(8\) \\
\hline
\(S_p(n)\) & \(1,82\) & \(2,50\) & \(3,08\) & \(3,57\) & \(4,00\) & \(4,38\) & \(4,71\) \\
\hline
\(f\) & \(0,10\) & \(0,10\) & \(0,10\) & \(0,10\) & \(0,10\) & \(0,10\) & \(0,10\) \\
\hline
\end{tabular}
\caption{Dane dla przykładu \ref{ex:karp1}.}
\label{tab:karp_flat1}
\end{table}

Ponieważ eksperymentalnie wyznaczona część sekwencyjna algorytmu nie rośnie wraz z ilością procesorów, możemy wnioskować, że zbyt duża część obliczeń jest inherentnie sekwencyjna.
\end{przyklad}

\begin{przyklad}\label{ex:karp2}
Powiedzmy, że testując algorytm równoległy na \(1, 2, \dots, 8\) procesorach otrzymaliśmy pewne przyspieszenia w zależności od ilości procesów i z wyrażenia \ref{eq:karp_flatt_3} obliczyliśmy eksperymentalnie wyznaczoną część sekwencyjną f. Dane zebrane są w tablicy \ref{tab:karp_flat2}.

\begin{table}[H]
\centering
\begin{tabular}{|c|c|c|c|c|c|c|c|}
\hline
\(p\) & \(2\) & \(3\) & \(4\) & \(5\) & \(6\) & \(7\) & \(8\) \\
\hline
\(S_p(n)\) & \(1,87\) & \(2,61\) & \(3,23\) & \(3,73\) & \(4,14\) & \(4,46\) & \(4,71\) \\
\hline
\(f\) & \(0,070\) & \(0,075\) & \(0,080\) & \(0,085\) & \(0,090\) & \(0,095\) & \(0,10\) \\
\hline
\end{tabular}
\caption{Dane dla przykładu \ref{ex:karp2}.}
\label{tab:karp_flat2}
\end{table}
Ponieważ eksperymentalnie wyznaczona część sekwencyjna algorytmu rośnie wraz z ilością procesorów, możemy wnioskować, że przyczyną niskiego przyspieszenia jest organizacja obliczeń równoległych, tj. czas poświęcony uruchomieniu procesów, komunikacji między nimi, synchronizacji lub ograniczenia samej architektury.
\end{przyklad}