\subsection{Złożonośc czasowa}
\label{subsec:algorytmy_sekwencyjne}
\subsubsection{Algorytmy sekwencyjne}

Ograniczenia zasobów (np. czasu i przestrzeni) wymagane przez algorytmy sekwencyjne mierzymy jako funkcję rozmiaru danych wejściowych \(T(n)\), tzw. złożoność czasową. Ograniczenia te wyrażamy asymptotycznie używając notacji:

\begin{enumerate}
\item{\(T(n) = O(f(n))\), jeśli istnieje dodatnie stałe \(c\) i \(n_0\) takie, że \(\forall{n \geq n_0}: (T(n)\leq cf(n)) \)}
\item{\(T(n) = \Omega(f(n))\), jeśli istnieje dodatnie stałe \(c\) i \(n_0\) takie, że \(\forall{n \geq n_0}: (T(n)\geq cf(n)) \)}
\item{\(T(n) = \Theta(f(n))\), jeśli \(T(n)=O(f(n))\) i \(T(n)=\Omega(f(n))\)}
\end{enumerate}
Czas działania algorytmu sekwencyjnego szacuje się przez liczbę operacji podstawowych wymaganych przez algorytm jako funkcję ilości danych wejściowych.
\subsubsection{Algorytmy równoległe}

\begin{definicja}[Pesymistyczna złożoność obliczeniowa\cite{Czech}]\label{def:pesymistyczna_zlozonosc_czasowa}
Załóżmy że algorytm równoległy \(R\) rozwiązuje problem \(P\) o rozmiarze \(n\). \textbf{Pesymityczną złożonością czasową algorytmu} \(R\) nazywamy funkcję:\\
\begin{align}
T_{p}(n) = \sup_{d\in{D_n}}{\left\{t(p,d)\right\}},
\end{align}
gdzie \(t(p,d)\) oznacza liczbę kroków obliczeniowych (operacji dominujących) wykonanych dla zestawu danych \(d\) od momentu rozpoczęcia obliczeń algorytmu \(R\) przez pierwszy procesor do chwili zakończenia obliczeń przez wszystkie procesory, \(p\) -- liczbę procesorów, \(D_n\) -- zbiór wszystkich zestawów danych wejściowych \(d\) o rozmiarze \(n\).
\end{definicja}